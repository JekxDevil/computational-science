\documentclass[unicode,11pt,a4paper,oneside,numbers=endperiod,openany, draft]{scrartcl}

% Required package
\usepackage{amssymb}
\usepackage{graphicx}
\usepackage{amsmath}
\usepackage{matlab-prettifier}
\usepackage{float}
\usepackage[export]{adjustbox}
\usepackage{multirow}
\usepackage{booktabs}
\usepackage{amsthm} % math theorems
\usepackage{ifthen}
\usepackage{physics} % responsive norm, abs, ...
\usepackage{algorithm}
\usepackage{algpseudocode}


\renewcommand{\thesubsection}{\arabic{subsection}}

% 1: command name, 2: Title, 3: subtitle, 4: label, 5: content
\newcommand{\mytheorem}[5]{\newtheorem*{#1}{#2} \begin{#1}[#3]\label{#4} #5 \end{#1}}

% 1: if numbered equation, 2: label, 3: content
\newcommand{\myex}[3]{
    \ifthenelse{\equal{#1}{true}}{
        \begin{equation} \label{#2} \begin{aligned} #3 \end{aligned} \end{equation}
    }{
        \begin{equation*} \label{#2} \begin{aligned} #3 \end{aligned} \end{equation*}
    }
}

% vector shortcut
\newcommand{\myvec}[1]{\begin{bmatrix} #1 \end{bmatrix}}

% 1: caption, 2: label, 3: trim, 4: figure within figures folder
\newcommand{\myfigure}[4]{
    \begin{figure}[htbp]
    \centering
    \caption{#1}
    \label{#2}
    \includegraphics[width=\paperwidth, trim=#3]{./figures/#4}
    \end{figure}
}

\input{assignment.sty}
\begin{document}

\setassignment
\setduedate{Monday, 3 June 2024, 12:00 AM}

\serieheader
{Optimization Methods}
{2024}
{\textbf{Student:} Jeferson Morales Mariciano \\\\}
{\textbf{Discussed with:} }
{Assignment 4}{}
\newline

%----------------------------------------------------------------------------------
\section{Exercise (20/100)}

Consider the quadratic function \( f: \mathbb{R}^2 \rightarrow \mathbb{R} \) defined as:

\myex{true}{eq:ex1-f}{
    f(\mathbf{x}) = 7 x^2 + 4 xy + y^2
}

where \( \mathbf{x} = (x, y)^T \).

\begin{enumerate}
    \item Write this function in canonical form, i.e. 
    \( f( \mathbf{x} ) = \frac{1}{2} \mathbf{x}^T \mathbf{A} \mathbf{x} + \mathbf{b}^T \mathbf{x} + c \),
    where \( A \) is a symmetric matrix.

    \item Descrive briefly how th Conjugate Gradient (CG) Method works 
    and discuss whether it is suitable to minimize \( f \) from equation \ref{eq:ex1-f}.
    Explain your reasoning in detail (max. 30 lines).
\end{enumerate}



%----------------------------------------------------------------------------------
\section{Exercise (20/100)}

Consider the following constrained minimization problem for \( \mathbf{x} = (x, y, z)^T \)

\myex{true}{eq:ex2-min-f}{
    &\min_{\mathbf{x}} f(\mathbf{x}) := -3x^2 + y^2 + 2z^2 + 2(x + y + z) \\
    &\text{subject to} \quad c(\mathbf{x}) = x^2 + y^2 + z^2 - 1 = 0
}

Write down the Lagrangian function and derive th KKT conditions for \ref{eq:ex2-min-f}




%----------------------------------------------------------------------------------
\section{Exercise (60/100)}

\begin{enumerate}
    \item Read the chapter on Simplex method, in particular the section 13.3 The Simplex Method, 
    in Numerical Optimization, Nocedal and Wright. 
    Explain how the method works, with a particular attention to the search direction.

    \item Consider the following contrained minimization problem,
    \( \mathbf{x} = (x_1, x_2)^T \);

    \myex{true}{eq:ex3-min-f}{
        \min_{\mathbf{x}} f(\mathbf{x}) := 4x_1 + 3x_2
    }

    subect to:

    \myex{true}{eq:ex3-constraints}{
        6 - 2x_1 - 3x_2 \geq 0 \\
        3 + 3x_1 - 2x_2 \geq 0 \\
        5 - 2x_2 \geq 0 \\
        4 - 2x_1 - x_2 \geq 0 \\
        x_2 \geq 0 \\
        x_1 \geq 0 \\
    }

    \begin{enumerate}
        \item Sketch the feasible region for this problem.

        \item Which are the basic feasible points of the problem \ref{eq:ex3-min-f}? 
        Compute them by hand using the geometrical interpretation 
        and find the optimal point \( \mathbf{x^*} \) 
        that minimizes \( f \) subject to the constraints.

        \item Prove that the first order necessary conditions holds for the optimal point.
    \end{enumerate}
\end{enumerate}

\end{document}

\documentclass[unicode,11pt,a4paper,oneside,numbers=endperiod,openany]{scrartcl}

% Required package
\usepackage{amssymb}
\usepackage{graphicx}
\usepackage{amsmath}
\usepackage{matlab-prettifier}
\usepackage{float}
\usepackage[export]{adjustbox}
\usepackage{multirow}
\usepackage{booktabs}
\usepackage{amsthm} % math theorems
\usepackage{ifthen}
\usepackage{physics} % responsive norm, abs, ...
\usepackage{algorithm}
\usepackage{algpseudocode}
\usepackage{array} % for table
\usepackage{mathtools} % for \coloneq
\usepackage{xcolor, soul} % highlight
\usepackage{enumitem} % custom enumeration in lists

\renewcommand{\thesubsection}{\arabic{subsection}}

\newtheorem{theorem}{Theorem}[section]
% 1: title, 2: label, 3: content
\newcommand{\myth}[3]{
    \begin{theorem}[#1] 
        \label{#2} 
        #3 
    \end{theorem}
}

% 1: color, 2: content
\newcommand{\mathcolorbox}[2]{\colorbox{#1}{\(\displaystyle #2\)}}

% 1: if numbered equation, 2: label, 3: content
\newcommand{\myex}[3]{
    \ifthenelse{\equal{#1}{true}}{
        \begin{equation} \label{#2} \begin{aligned} #3 \end{aligned} \end{equation}
    }{
        \begin{equation*} \label{#2} \begin{aligned} #3 \end{aligned} \end{equation*}
    }
}

% vector shortcut
\newcommand{\myvec}[1]{\begin{bmatrix} #1 \end{bmatrix}}

% 1: letter of vector, 2: 
\newcommand\vibar[2]{\mathbf{\overline{#1}}_#2}

% 1: caption, 2: label, 3: trim, 4: figure within figures folder
\newcommand{\myfigure}[4]{
    \begin{figure}[htbp]
    \centering
    \caption{#1}
    \label{#2}
    \includegraphics[width=\paperwidth, trim=#3]{./figures/#4}
    \end{figure}
}

\def\ex2f{f(\mathbf{x}^*)}

\input{assignment.sty}
\begin{document}

\setassignment
\setduedate{Thursday, 10 October 2024, 23:59}

\serieheader
{Discrete Mathematics}
{2024}
{%
\textbf{Student:} Jeferson Morales Mariciano 
\href{mailto:jmorale@ethz.ch}{\(<\)jmorale@ethz.ch\(>\)} \\\\}
{\vspace{-1cm}}%\textbf{Discussed with:} }
{Assignment 3}{}

%----------------------------------------------------------------------------------
\section*{Exercise 3.2, From Natural Language to a Formula (\(\star\)) \hfill (4 Points)}

Consider the universe \(U = \mathbb{N} \setminus\{0\}\). 
Express each of the following statements with a formula in predicate logic, 
in which the only predicates appearing are 
\(divides(x, y)\), \(equals(x, y)\) and \(prime(x)\) 
(instead of divides(x, y) and equals(x, y) you can write \(x | y\) and \(x = y\) accordingly). 
You can also use the symbols \(+\) and \(cdot\) 
to denote the addition and multiplication functions, 
and you can use constants (e.g., 0, 1, . . .). 
You can also use \(\Leftrightarrow\). 
No justification is required.

\begin{enumerate}[label=(\roman*)]
    \item \((\star)\) If a number divides two numbers, then it also divides their sum.
    \item \((\star)\) The only divisors of a prime number are 1 and the number itself.
    \item \((\star)\) 1 is the only natural number which has an inverse.
    \item \((\star)\) A prime number divides the product of two natural numbers 
        if and only if it divides at least one of them.
\end{enumerate}



\section*{Exercise 3.8, Proof by Contradiction (\(\star\)) \hfill (4 Points)}

Let \(n, m \in \mathbb{N}\) be arbitrary. 
We say “n divides m” and write \(n | m\) 
if there exists a \(k \in \mathbb{N}\) such that \(k \cdot n = m\). 
Prove that the following statement is true, 
using a proof by contradiction:
\[
n | m \;\text{ and }\; n |(m + 1) \Longrightarrow n = 1.
\]
You are allowed to invoke the statement 3.2 \textbf{iii)} from above to justify one step.
You must use the same notation as in the lecture notes, 
i.e. precisely state what your statements S and T are, 
and justify each of your proof steps.

\end{document}

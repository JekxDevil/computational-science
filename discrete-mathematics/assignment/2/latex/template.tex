\documentclass[unicode,11pt,a4paper,oneside,numbers=endperiod,openany]{scrartcl}

% Required package
\usepackage{amssymb}
\usepackage{graphicx}
\usepackage{amsmath}
\usepackage{matlab-prettifier}
\usepackage{float}
\usepackage[export]{adjustbox}
\usepackage{multirow}
\usepackage{booktabs}
\usepackage{amsthm} % math theorems
\usepackage{ifthen}
\usepackage{physics} % responsive norm, abs, ...
\usepackage{algorithm}
\usepackage{algpseudocode}
\usepackage{array} % for table
\usepackage{mathtools} % for \coloneq
\usepackage{xcolor, soul} % highlight

\renewcommand{\thesubsection}{\arabic{subsection}}

\newtheorem{theorem}{Theorem}[section]
% 1: title, 2: label, 3: content
\newcommand{\myth}[3]{
    \begin{theorem}[#1] 
        \label{#2} 
        #3 
    \end{theorem}
}

% 1: color, 2: content
\newcommand{\mathcolorbox}[2]{\colorbox{#1}{\(\displaystyle #2\)}}

% 1: if numbered equation, 2: label, 3: content
\newcommand{\myex}[3]{
    \ifthenelse{\equal{#1}{true}}{
        \begin{equation} \label{#2} \begin{aligned} #3 \end{aligned} \end{equation}
    }{
        \begin{equation*} \label{#2} \begin{aligned} #3 \end{aligned} \end{equation*}
    }
}

% vector shortcut
\newcommand{\myvec}[1]{\begin{bmatrix} #1 \end{bmatrix}}

% 1: letter of vector, 2: 
\newcommand\vibar[2]{\mathbf{\overline{#1}}_#2}

% 1: caption, 2: label, 3: trim, 4: figure within figures folder
\newcommand{\myfigure}[4]{
    \begin{figure}[htbp]
    \centering
    \caption{#1}
    \label{#2}
    \includegraphics[width=\paperwidth, trim=#3]{./figures/#4}
    \end{figure}
}

\def\ex2f{f(\mathbf{x}^*)}

\input{assignment.sty}
\begin{document}

\setassignment
\setduedate{Thursday, 3 October 2024, 23:59}

\serieheader
{Discrete Mathematics}
{2024}
{%
\textbf{Student:} Jeferson Morales Mariciano 
\href{mailto:jmorale@ethz.ch}{\(<\)jmorale@ethz.ch\(>\)} \\\\}
{\vspace{-1cm}}%\textbf{Discussed with:} }
{Assignment 2}{}

%----------------------------------------------------------------------------------
\section*{Exercise 2.3, Simplifying a Formula (\(\star\)) \hfill (8 Points)}

Consider the propositional formula

\[
F = \left( (B \lor C) \to ((A \lor \neg B) \land C) \right) \lor (A \land \neg C)
\]

\noindent Give a formula \( G \) that is equivalent to \( F \), but in which each atomic formula \( A \), \( B \), and \( C \)
appears at most once. Prove that \( F \equiv G \) by providing a sequence of equivalence transformations 
with \textit{at most} 12 steps.

\noindent \textbf{Expectation.} 
Your proof should be in the form of a sequence of steps, where each step
consists of applying the definition of \( \to \) 
(that is \( F \to G \equiv \neg F \lor G \)), 
one of the rules given in Lemma 2.1 of the lecture notes 
\footnote{Lemma 2.1 states rules involving propositional symbols, 
but you may apply those rules at the level of formulas 
(see Section 2.3.5 of the lecture notes).}, 
or one of the following rules: 
\( F \land \neg F \equiv \bot \), 
\( F \land \bot \equiv \bot \), 
\( F \lor \bot \equiv F \), 
\( F \lor \neg F \equiv \top \), 
\( F \land \top \equiv F \), 
and \( F \lor \top \equiv \top \). 
For this exercise, associativity is to be applied as in Lemma 2.1.3. 
Each step of your proof should apply a \textit{single} rule \textit{once} 
and state \textit{which} rule was applied.
\\

The formula \( G \) equivalent to \( F \) is given by \( G = \neg B \lor A \), 
where each propositional symbol appears at most once.
I provide two proofs (\ref{ex2-3-proof}, \ref{ex2-3-proof-rigorous}), 
the second one (\ref{ex2-3-proof-rigorous}) is a rigorous explanation of all correct steps done to follow
\textbf{exactly} what Lemma 2.1 and the before-mentioned properties state. 
The first proof (\ref{ex2-3-proof}) presented below is 12-steps long as requested: 
it uses commutativity only to put expressions nearby to apply the rules. 

\myex{true}{ex2-3-proof}{
F &= \left( 
        \left(B \lor C \right) 
        \rightarrow 
        \left( 
            \left(A \lor \neg B \right) 
            \land C 
        \right) 
    \right) 
    \lor \left( A \land \neg C \right) \\
&\equiv \left( 
        \mathcolorbox{yellow}{
            \neg \left(B \lor C \right) \lor \left( \left( A \lor \neg B \right) \land C \right)
        } 
    \right) 
    \lor \left( A \land \neg C \right) 
&&&&&&&&&& \text{def. of implication \(\rightarrow\)} \\
&\equiv \left( 
        \left(
            \mathcolorbox{yellow}{ \neg B \land \neg C} 
        \right) 
        \lor 
        \left( 
            \left(A \lor \neg B \right) 
            \land C 
        \right) 
    \right) 
    \lor \left( A \land \neg C \right)
&&&&&&&&&& \text{def. De Morgan rule} \\
&\equiv \left( 
        \left( \neg B \land \neg C \right) 
        \lor \left( 
            \mathcolorbox{yellow}{\left( C \land A \right) \lor \left( C \land \neg B \right)}
        \right)
    \right)
    \lor \left( A \land \neg C \right)
&&&&&&&&&& \text{def. 1st distributive law} \\
&\equiv \left( 
        \left( \neg B \land \neg C \right) 
        \lor \left( 
            \mathcolorbox{yellow}{\left( C \land \neg B \right) \lor \left( C \land A \right)}
        \right)
    \right)
    \lor \left( A \land \neg C \right)
&&&&&&&&&& \text{def. commutative of } \lor \\
&\equiv  % comment me
    \left( 
        \left( 
            \mathcolorbox{yellow}{
                \left( \neg B \land \neg C \right) 
                \lor  
                \left( C \land \neg B \right)
            } 
        \right) 
        \lor \left( C \land A \right)
    \right)
    \lor \left( A \land \neg C \right)
&&&&&&&&&& \text{def. associativity of } \lor \\ % until here
&\equiv  % comment me
    \left( 
        \left(
            \mathcolorbox{yellow}{
                \neg B \land \left( \neg C \lor C \right)   
            }
        \right)
        \lor \left( C \land A \right)
    \right)
    \lor \left( A \land \neg C \right)
&&&&&&&&&& \text{def. 1st distributive law} \\
&\equiv 
    \left( 
        \left(
            \mathcolorbox{yellow}{
                \neg B \land \top  
            }
        \right)
        \lor \left( C \land A \right)
    \right)
    \lor \left( A \land \neg C \right)
&&&&&&&&&& F \lor \neg F \equiv \top \\
&\equiv \left( 
        \mathcolorbox{yellow}{\neg B}
        \lor \left( C \land A \right)
    \right)
    \lor \left( A \land \neg C \right)
&&&&&&&&&& F \land \top \equiv F \\
&\equiv \neg B 
    \lor \left(
    \mathcolorbox{yellow}{
        \left( C \land A \right) \lor \left( A \land \neg C \right)
    }
    \right)
&&&&&&&&&& \text{def. associativity of } \lor \\
}
\myex{false}{ex2-3-proof-2}{
&\equiv \neg B 
    \lor \left(
    \mathcolorbox{yellow}{
        A \land \left( C \lor \neg C \right)
    }
    \right)
&&&&&&&&&& \text{def. 1st distributive law}  \\
&\equiv \neg B 
    \lor \left(
    \mathcolorbox{yellow}{
        A \land \top
    }
    \right)
&&&&&&&&&& F \lor \neg F \equiv \top \\
&\equiv \neg B \lor \mathcolorbox{yellow}{A}
&&&&&&&&&& F \land \top \equiv F \\
}


Finally, by applying a 16-step sequence of rigorous equivalence transformations
as shown in Lemma 2.1 and in the assignment, 
it is shown \( F \equiv G \).
It is not possible to simplify the formula to 12 steps 
and still be able to perform the equivalence transformations
with the same order and precedence as reported in Lemma 2.1 and here-mentioned properties.

% rigorous proof =========================================================================
\myex{true}{ex2-3-proof-rigorous}{
F &= \left( 
        \left(B \lor C \right) 
        \rightarrow 
        \left( 
            \left(A \lor \neg B \right) 
            \land C 
        \right) 
    \right) 
    \lor \left( A \land \neg C \right) \\
&\equiv \left( 
        \mathcolorbox{yellow}{
            \neg \left(B \lor C \right) 
            \lor \left( \left( A \lor \neg B \right) \land C \right)
        } 
    \right) 
    \lor \left( A \land \neg C \right) 
&&&&&&&&&& \text{def. of implication \(\rightarrow\)} \\
&\equiv \left( 
        \left(
            \mathcolorbox{yellow}{ \neg B \land \neg C} 
        \right) 
        \lor 
        \left( 
            \left(A \lor \neg B \right) 
            \land C 
        \right) 
    \right) 
    \lor \left( A \land \neg C \right)
&&&&&&&&&& \text{def. De Morgan rule} \\
&\equiv \left( 
        \left( \neg B \land \neg C \right) 
        \lor \left( \mathcolorbox{yellow}{ C \land \left( A \lor \neg B \right)} \right)
    \right)
    \lor \left( A \land \neg C \right)
&&&&&&&&&& \text{def. commutative of } \land \\
&\equiv \left( 
        \left( \neg B \land \neg C \right) 
        \lor \left( 
            \mathcolorbox{yellow}{\left( C \land A \right) \lor \left( C \land \neg B \right)}
        \right)
    \right)
    \lor \left( A \land \neg C \right)
&&&&&&&&&& \text{def. 1st distributive law} \\
&\equiv \left( 
        \left( \neg B \land \neg C \right) 
        \lor \left( 
            \mathcolorbox{yellow}{\left( C \land \neg B \right) \lor \left( C \land A \right)}
        \right)
    \right)
    \lor \left( A \land \neg C \right)
&&&&&&&&&& \text{def. commutative of } \lor \\
&\equiv  % comment me
    \left( 
        \left( 
            \mathcolorbox{yellow}{
                \left( \neg B \land \neg C \right) 
                \lor  
                \left( C \land \neg B \right)
            } 
        \right) 
        \lor \left( C \land A \right)
    \right)
    \lor \left( A \land \neg C \right)
&&&&&&&&&& \text{def. associativity of } \lor \\ % until here
&\equiv 
    \left( %comment me
        \left(
            \left( \neg B \land \neg C \right) 
            \lor \left( \mathcolorbox{yellow}{  \neg B \land C } \right) 
        \right)
        \lor \left( C \land A \right)
    \right)
    \lor \left( A \land \neg C \right)
&&&&&&&&&& \text{def. commutative of } \land \\ % until here
&\equiv  % comment me
    \left( 
        \left(
            \mathcolorbox{yellow}{
                \neg B \land \left( \neg C \lor C \right)   
            }
        \right)
        \lor \left( C \land A \right)
    \right)
    \lor \left( A \land \neg C \right)
&&&&&&&&&& \text{def. 1st distributive law} \\
&\equiv 
    \left( 
        \left(
            \neg B \land \left( \mathcolorbox{yellow}{ C \lor \neg C } \right)   
        \right)
        \lor \left( C \land A \right)
    \right)
    \lor \left( A \land \neg C \right)
&&&&&&&&&& \text{def. commutative of } \lor \\
&\equiv 
    \left( 
        \left(
            \mathcolorbox{yellow}{
                \neg B \land \top  
            }
        \right)
        \lor \left( C \land A \right)
    \right)
    \lor \left( A \land \neg C \right)
&&&&&&&&&& F \lor \neg F \equiv \top \\
&\equiv \left( 
        \mathcolorbox{yellow}{\neg B}
        \lor \left( C \land A \right)
    \right)
    \lor \left( A \land \neg C \right)
&&&&&&&&&& F \land \top \equiv F \\
&\equiv \neg B 
    \lor \left(
    \mathcolorbox{yellow}{
        \left( C \land A \right) \lor \left( A \land \neg C \right)
    }
    \right)
&&&&&&&&&& \text{def. associativity of } \lor \\
&\equiv \neg B % comment me
    \lor 
    \left(
        \left(
            \mathcolorbox{yellow}{ A \land C }
        \right)
        \lor 
        \left( A \land \neg C \right)
    \right)
&&&&&&&&&& \text{def. commutative of } \land \\ %until here
&\equiv \neg B 
    \lor \left(
    \mathcolorbox{yellow}{
        A \land \left( C \lor \neg C \right)
    }
    \right)
&&&&&&&&&& \text{def. 1st distributive law}  \\
&\equiv \neg B 
    \lor \left(
    \mathcolorbox{yellow}{
        A \land \top
    }
    \right)
&&&&&&&&&& F \lor \neg F \equiv \top \\
&\equiv \neg B \lor \mathcolorbox{yellow}{A}
&&&&&&&&&& F \land \top \equiv F \\
}

\end{document}

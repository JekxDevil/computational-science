\documentclass[unicode,11pt,a4paper,oneside,numbers=endperiod,openany]{scrartcl}

% Required package
\usepackage{amssymb}
\usepackage{graphicx}
\usepackage{amsmath}
\usepackage{matlab-prettifier}
\usepackage{float}
\usepackage[export]{adjustbox}
\usepackage{multirow}
\usepackage{booktabs}
\usepackage{amsthm} % math theorems
\usepackage{ifthen}
\usepackage{physics} % responsive norm, abs, ...
\usepackage{algorithm}
\usepackage{algpseudocode}
\usepackage{array} % for table
\usepackage{mathtools} % for \coloneq
\usepackage{xcolor, soul} % highlight
\usepackage{enumitem} % custom enumeration in lists
\usepackage{dsfont} % 1 for identity matrix
\usepackage{blkarray, bigstrut} % block matrix with labels
\usepackage{tikz} % for drawing

\newcommand\topstrut[1][1.2ex]{\setlength\bigstrutjot{#1}{\bigstrut[t]}}
\newcommand\botstrut[1][0.9ex]{\setlength\bigstrutjot{#1}{\bigstrut[b]}}

\usetikzlibrary{positioning}
\tikzset{
    > = stealth,
    every path/.append style = {
        arrows = -,
    }
}

\renewcommand{\thesubsection}{\arabic{subsection}}
\def\N{\mathbb{N}}
\def\Z{\mathbb{Z}}
\def\Zp{\mathbb{Z}^+}
\def\gcd{\textrm{gcd}}
\def\lcm{\textrm{lcm}}
\newcommand{\Zgmult}[1]{\mathbb{Z}_{#1}^{\ast}}


\newtheorem{theorem}{Theorem}[section]
% 1: title, 2: label, 3: content
\newcommand{\myth}[3]{
    \begin{theorem}[#1] 
        \label{#2} 
        #3 
    \end{theorem}
}

% 1: color, 2: content
\newcommand{\mathcolorbox}[2]{\colorbox{#1}{\(\displaystyle #2\)}}

% 1: if numbered equation, 2: label, 3: content
\newcommand{\myex}[3]{
    \ifthenelse{\equal{#1}{true}}{
        \begin{equation} \label{#2} \begin{aligned} #3 \end{aligned} \end{equation}
    }{
        \begin{equation*} \label{#2} \begin{aligned} #3 \end{aligned} \end{equation*}
    }
}

% vector shortcut
\newcommand{\myvec}[1]{\begin{bmatrix} #1 \end{bmatrix}}

% 1: letter of vector, 2: 
\newcommand\vibar[2]{\mathbf{\overline{#1}}_#2}

% 1: caption, 2: label, 3: trim, 4: figure within figures folder
\newcommand{\myfigure}[4]{
    \begin{figure}[htbp]
    \centering
    \caption{#1}
    \label{#2}
    \includegraphics[width=\paperwidth, trim=#3]{./figures/#4}
    \end{figure}
}

% proof step
\newcommand{\pstep}{\overset{.}{\Longrightarrow}}

\def\ex2f{f(\mathbf{x}^*)}

\input{assignment.sty}
\begin{document}

\setassignment
\setduedate{Thursday, 21 November 2024, 23:59}

\serieheader
{Discrete Mathematics}
{2024}
{%
\textbf{Student:} Jeferson Morales Mariciano 
\href{mailto:jmorale@ethz.ch}{\(<\)jmorale@ethz.ch\(>\)} \\\\}
{\vspace{-1cm}}%\textbf{Discussed with:} }
{Assignment 9}{}

%----------------------------------------------------------------------------------
\section*{Exercise 9.5, More Elementary Properties of Rings \( (\star \star) \) \hfill (8 Points)}
\textbf{Note:} in a previous version of this exercise the assumption that 
\( R \) is an integral domain was missing.
However, the first statement is false in this case.
Can you give a counterexample? \newline
\noindent Let \( \langle R; +, -, 0, \cdot, 1 \rangle \) be a ring,
and let \( a \in R \) and \( b \in R \). 
Prove the following statements:

\begin{enumerate}[label=\textbf{\alph*)}]
    \item 
    If \( R \) is an integral domain and if \( a^m = b^m \) and \( a^n = b^n \) 
    for some positive integers \( m \) and \( n \) with \( \gcd(m,n) = 1 \),
    then \( a = b \).

    \item 
    If \( 1 - ab \) is a unit, then \( 1 - ba \) is also a unit. 
    \textit{Hint: if \( x = (1 - ab)^{-1} \) consider the ring element \( 1 + bxa \)}.

\end{enumerate}

\subsection*{a)}

%----------------------------------------------------------------------------------
%----------------------------------------------------------------------------------
%----------------------------------------------------------------------------------
\subsection*{b)}


\end{document}

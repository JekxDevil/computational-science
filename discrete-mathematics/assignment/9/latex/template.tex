\documentclass[unicode,11pt,a4paper,oneside,numbers=endperiod,openany]{scrartcl}

% Required package
\usepackage{amssymb}
\usepackage{graphicx}
\usepackage{amsmath}
\usepackage{matlab-prettifier}
\usepackage{float}
\usepackage[export]{adjustbox}
\usepackage{multirow}
\usepackage{booktabs}
\usepackage{amsthm} % math theorems
\usepackage{ifthen}
\usepackage{physics} % responsive norm, abs, ...
\usepackage{algorithm}
\usepackage{algpseudocode}
\usepackage{array} % for table
\usepackage{mathtools} % for \coloneq
\usepackage{xcolor, soul} % highlight
\usepackage{enumitem} % custom enumeration in lists
\usepackage{dsfont} % 1 for identity matrix
\usepackage{blkarray, bigstrut} % block matrix with labels
\usepackage{tikz} % for drawing

\newcommand\topstrut[1][1.2ex]{\setlength\bigstrutjot{#1}{\bigstrut[t]}}
\newcommand\botstrut[1][0.9ex]{\setlength\bigstrutjot{#1}{\bigstrut[b]}}

\usetikzlibrary{positioning}
\tikzset{
    > = stealth,
    every path/.append style = {
        arrows = -,
    }
}

\renewcommand{\thesubsection}{\arabic{subsection}}
\def\N{\mathbb{N}}
\def\Z{\mathbb{Z}}
\def\Zp{\mathbb{Z}^+}
\def\gcd{\textrm{gcd}}
\def\lcm{\textrm{lcm}}
\newcommand{\Zgmult}[1]{\mathbb{Z}_{#1}^{\ast}}
\def\Ring{R}
\def\RingUnits{R^*}


\newtheorem{theorem}{Theorem}[section]
% 1: title, 2: label, 3: content
\newcommand{\myth}[3]{
    \begin{theorem}[#1] 
        \label{#2} 
        #3 
    \end{theorem}
}

% 1: color, 2: content
\newcommand{\mathcolorbox}[2]{\colorbox{#1}{\(\displaystyle #2\)}}

% 1: if numbered equation, 2: label, 3: content
\newcommand{\myex}[3]{
    \ifthenelse{\equal{#1}{true}}{
        \begin{equation} \label{#2} \begin{aligned} #3 \end{aligned} \end{equation}
    }{
        \begin{equation*} \label{#2} \begin{aligned} #3 \end{aligned} \end{equation*}
    }
}

% vector shortcut
\newcommand{\myvec}[1]{\begin{bmatrix} #1 \end{bmatrix}}

% 1: letter of vector, 2: 
\newcommand\vibar[2]{\mathbf{\overline{#1}}_#2}

% 1: caption, 2: label, 3: trim, 4: figure within figures folder
\newcommand{\myfigure}[4]{
    \begin{figure}[htbp]
    \centering
    \caption{#1}
    \label{#2}
    \includegraphics[width=\paperwidth, trim=#3]{./figures/#4}
    \end{figure}
}

% proof step
\newcommand{\pstep}{\overset{.}{\Longrightarrow}}
\newcommand{\spstep}{\overset{.}{\Longleftrightarrow}}

\def\ex2f{f(\mathbf{x}^*)}

\input{assignment.sty}
\begin{document}

\setassignment
\setduedate{Thursday, 21 November 2024, 23:59}

\serieheader
{Discrete Mathematics}
{2024}
{%
\textbf{Student:} Jeferson Morales Mariciano 
\href{mailto:jmorale@ethz.ch}{\(<\)jmorale@ethz.ch\(>\)} \\\\}
{\vspace{-1cm}}%\textbf{Discussed with:} }
{Assignment 9}{}

%----------------------------------------------------------------------------------
\section*{Exercise 9.5, More Elementary Properties of Rings \( (\star \star) \) \hfill (8 Points)}
\textbf{Note:} in a previous version of this exercise the assumption that 
\( R \) is an integral domain was missing.
However, the first statement is false in this case.
Can you give a counterexample? \newline
\noindent Let \( \langle R; +, -, 0, \cdot, 1 \rangle \) be a ring,
and let \( a \in R \) and \( b \in R \). 
Prove the following statements:

\begin{enumerate}[label=\textbf{\alph*)}]
    \item 
    If \( R \) is an integral domain and if \( a^m = b^m \) and \( a^n = b^n \) 
    for some positive integers \( m \) and \( n \) with \( \gcd(m,n) = 1 \),
    then \( a = b \).

    \item 
    If \( 1 - ab \) is a unit, then \( 1 - ba \) is also a unit. 
    \textit{Hint: if \( x = (1 - ab)^{-1} \) consider the ring element \( 1 + bxa \)}.

\end{enumerate}

\subsection*{a)}
Assumption: \( R \) is an integral domain and 
    \( \exists m, n \in \Z^+, \; a^m = b^m \; \land \; a^n = b^n \; \land \; \gcd(m,n) = 1 \). 
    \newline
Claim: \( a = b \). \newline
Proof: \newline
Distinguish two main cases for the proof, where \( a = 0 \) and \( a \neq 0 \). \newline
Thanks to the integral domain property, 
whenever \( a = 0 \), then also \( b = 0 \) since \( 0^m = 0^n = 0 \). \newline
For \( a \neq 0 \), 
recall from Corollary 4.5, for \( a, b \in \Z \) not both 0, \( \exists u, v \in \Z \) such that
\( \gcd(a, b) = ua + vb \).
\( \gcd(a, b) = 1 = um + vn \) , where \( m, n \in \Z^+ \) 
since the assumption states they are positive integers. \newline
It means that either \( u < 0 \) or \( v < 0 \) but not both,
in order to comply for \( 1 = um + vn \). \newline
Without loss of generality (w.l.o.g.), assume \( u < 0 \) and \( v > 0 \).
Then, \( -u > 0 \).

\myex{false}{ex9-5-a}{
    a 
    & \; \spstep \; 
    a^1
    \\
    & \; \spstep \; 
    a^{um + vn}
    & \hspace{1cm} \hfill \text{(Corollary 4.5)}
    \\
    & \; \spstep \; 
    a^{um} \cdot a^{vn}
    & \hspace{1cm} \hfill \text{(multiplicative associativity)}
    \\
    & \; \spstep \; 
    (a^{m})^{u} \cdot (a^{n})^{v}
    & \hspace{1cm} \hfill \text{(multiplicative commutativity over \( \Z \))}
    \\
    & \; \spstep \; 
    (b^{m})^{u} \cdot (b^{n})^{v}
    & \hspace{1cm} \hfill \text{(assumption \( a^m = b^m, a^n = b^n \))}
    \\
    & \; \spstep \; 
    (b^{um}) \cdot (b^{vn})
    & \hspace{1cm} \hfill \text{(associativity and commutativity of \( \cdot \))}
    \\
    & \; \spstep \; 
    b^{um + vn} 
    & \hspace{1cm} \hfill \text{(multiplicative associativity)}
    \\
    & \; \spstep \; 
    b^{1} 
    & \hspace{1cm} \hfill \text{(Corollary 4.5)}
    \\
    & \; \spstep \; 
    b 
}

A symmetrical reasoning proves the case where \( u > 0 \) and \( v < 0 \).
The statement is therefore true and \( a = b \).

%----------------------------------------------------------------------------------
%----------------------------------------------------------------------------------
%----------------------------------------------------------------------------------
\subsection*{b)}

Assumption: \( \exists u_1 \in \RingUnits, u_1 = 1 - ab \), i.e. it is a unit. \newline
Claim: \( \exists u_2 \in \RingUnits, u_2 = 1 - ba \), it is also a unit. \newline
Proof: \newline
by Definition 5.20 of Units, \( u_1 \) is invertible,
meaning \( \exists v_1 \in \RingUnits,  u_1 \cdot v_1 = v_1 \cdot u_1 = 1 \). \newline
Let \( v_1 = (1 - ab)^{-1} \), be the multiplicative inverse of \( u_1 \). 
Then, \( u_1 \cdot v_1 = v_1 \cdot u_1 = 1 \). \newline 

From the Hint, consider the ring element \( w \in \Ring, w = 1 + b v_1 a \), 
where \( v_1 = (1 - ab)^{-1} \). \newline
In order for \( u_2 \) to be a unit,
check if it is invertible, 
i.e. \( \exists v_2 \in \RingUnits, u_2 \cdot v_2 = v_2 \cdot u_2 = 1 \). \newline 

Clearly, both \( v_2, w \in \Ring \), so let's check if \( u_2 \) is invertible
by assigning \( v_2 = w \). 

\myex{false}{ex9-5-b-proof}{
    u_2 \cdot v_2 
    & \; \pstep \; 
    (1 - ba) \cdot (1 + b v_1 a) 
    \\
    & \; \pstep \; 
    (1 - ba) \cdot 1 + (1-ba) \cdot b v_1 a 
    & \hspace{1cm} \hfill \text{(left distributivity)}
    \\
    & \; \pstep \; 
    1 - ba + b v_1 a - ba b v_1 a
    & \hspace{1cm} \hfill \text{(right distributivity)}
    \\
    & \; \pstep \; 
    1 - ba + b \cdot (v_1 a - ab v_1 a)
    & \hspace{1cm} \hfill \text{(left distributivity)}
    \\
    & \; \pstep \; 
    1 - ba + b \cdot ((1 - ab) \cdot v_1 a)
    & \hspace{1cm} \hfill \text{(right distributivity)}
    \\
    & \; \pstep \; 
    1 - ba + b \cdot ((1 - ab) \cdot (1 - ab)^{-1} \cdot a)
    & \hspace{1cm} \hfill \text{(def \( v_1 \))}
    \\
    & \; \pstep \; 
    1 - ba + b \cdot (1 \cdot a)
    & \hspace{1cm} \hfill \text{(assumption \( u_1 \cdot v_1 = 1 \) )} 
    \\
    & \; \pstep \; 
    1 - ba + ba
    & \hspace{1cm} \hfill \text{(multiplicative identity)}
    \\
    & \; \pstep \; 
    1
    & \hspace{1cm} \hfill \text{(abelian additive group inverse)} 
}

Symmetrically, \( v_2 \cdot u_2 = 1 \).

\myex{false}{ex9-5-b-proof}{
    v_2 \cdot u_2 
    & \; \pstep \; 
    (1 + b v_1 a) \cdot (1 - ba)  
    \\
    & \; \pstep \; 
    1 \cdot (1 - ba) + b v_1 a \cdot (1 - ba)
    & \hspace{1cm} \hfill \text{(right distributivity)}
    \\
    & \; \pstep \; 
    1 - ba + b v_1 a - b v_1 a ba
    & \hspace{1cm} \hfill \text{(left distributivity)}
    \\
    & \; \pstep \; 
    1 - ba + (b v_1 - b v_1 ab) \cdot a
    & \hspace{1cm} \hfill \text{(right distributivity)}
    \\
    & \; \pstep \; 
    1 - ba + (b v_1 \cdot (1 - ab)) \cdot a
    & \hspace{1cm} \hfill \text{(left distributivity)}
    \\
    & \; \pstep \; 
    1 - ba + (b \cdot (1 - ab)^{-1} \cdot (1 - ab)) \cdot a
    & \hspace{1cm} \hfill \text{(def \( v_1 \))}
    \\
    & \; \pstep \; 
    1 - ba + (b \cdot 1) \cdot a
    & \hspace{1cm} \hfill \text{(assumption \( v_1 \cdot u_1 = 1 \) )} 
    \\
    & \; \pstep \; 
    1 - ba + ba
    & \hspace{1cm} \hfill \text{(multiplicative identity)} 
    \\
    & \; \pstep \; 
    1
    & \hspace{1cm} \hfill \text{(abelian additive group inverse)} 
}

Thus, \( w \) satifies the definition of the inverse of the unit \( u_2 \),
therefore \( u_1, v_1, u_2, v_2 \in \RingUnits \), 
the statement is true and \( 1 - ba \) is also a unit. \newline




\end{document}

\documentclass[unicode,11pt,a4paper,oneside,numbers=endperiod,openany]{scrartcl}

% Required package
\usepackage{amssymb}
\usepackage{graphicx}
\usepackage{amsmath}
\usepackage{matlab-prettifier}
\usepackage{float}
\usepackage[export]{adjustbox}
\usepackage{multirow}
\usepackage{booktabs}
\usepackage{amsthm} % math theorems
\usepackage{ifthen}
\usepackage{physics} % responsive norm, abs, ...
\usepackage{algorithm}
\usepackage{algpseudocode}
\usepackage{array} % for table
\usepackage{mathtools} % for \coloneq
\usepackage{xcolor, soul} % highlight
\usepackage{enumitem} % custom enumeration in lists
\usepackage{dsfont} % 1 for identity matrix
\usepackage{blkarray, bigstrut} % block matrix with labels
\usepackage{tikz} % for drawing

\newcommand\topstrut[1][1.2ex]{\setlength\bigstrutjot{#1}{\bigstrut[t]}}
\newcommand\botstrut[1][0.9ex]{\setlength\bigstrutjot{#1}{\bigstrut[b]}}

\usetikzlibrary{positioning}
\tikzset{
    > = stealth,
    every path/.append style = {
        arrows = ->,
    }
}

\renewcommand{\thesubsection}{\arabic{subsection}}
\def\N{\mathbb{N}}
\def\Z{\mathbb{Z}}
\def\semiinfbinseq{\{0,1\}^\infty}

\newtheorem{theorem}{Theorem}[section]
% 1: title, 2: label, 3: content
\newcommand{\myth}[3]{
    \begin{theorem}[#1] 
        \label{#2} 
        #3 
    \end{theorem}
}

% 1: color, 2: content
\newcommand{\mathcolorbox}[2]{\colorbox{#1}{\(\displaystyle #2\)}}

% 1: if numbered equation, 2: label, 3: content
\newcommand{\myex}[3]{
    \ifthenelse{\equal{#1}{true}}{
        \begin{equation} \label{#2} \begin{aligned} #3 \end{aligned} \end{equation}
    }{
        \begin{equation*} \label{#2} \begin{aligned} #3 \end{aligned} \end{equation*}
    }
}

% vector shortcut
\newcommand{\myvec}[1]{\begin{bmatrix} #1 \end{bmatrix}}

% 1: letter of vector, 2: 
\newcommand\vibar[2]{\mathbf{\overline{#1}}_#2}

% 1: caption, 2: label, 3: trim, 4: figure within figures folder
\newcommand{\myfigure}[4]{
    \begin{figure}[htbp]
    \centering
    \caption{#1}
    \label{#2}
    \includegraphics[width=\paperwidth, trim=#3]{./figures/#4}
    \end{figure}
}

% proof step
\newcommand{\pstep}{\overset{.}{\Longrightarrow}}

\def\ex2f{f(\mathbf{x}^*)}

\input{assignment.sty}
\begin{document}

\setassignment
\setduedate{Thursday, 31 October 2024, 23:59}

\serieheader
{Discrete Mathematics}
{2024}
{%
\textbf{Student:} Jeferson Morales Mariciano 
\href{mailto:jmorale@ethz.ch}{\(<\)jmorale@ethz.ch\(>\)} \\\\}
{\vspace{-1cm}}%\textbf{Discussed with:} }
{Assignment 6}{}

%----------------------------------------------------------------------------------
\section*{Exercise 6.5, Countability \hfill (8 Points)}

Prove that for all \( l \in \N \) with \( l \geq 1 \) the set

\myex{false}{ex5-statement}{
    A_l \coloneq 
    \biggl\{
        f : \N \to \{0,1\} \bigg| \sum_{i=0}^{k} f(i) \leq \frac{k}{l} + 1 \quad \forall k \in \N
    \biggr\}.
}
is uncountable.\\
\textbf{Hint:} %Use Cantor's diagonal argument.
For all \( l \geq 1 \), explicitly write an injection 
from a known uncountable set into \( A_l \).
\\\newline

\noindent Following the hint, an injection from an uncountable set to \( A_l \) is going to be built.
Notice that to belong to \( A_l \; \forall l \, \geq 1 \), 
it is sufficient that the function \( f \) satisfies the following conditions:
yielding either always 0 for any input or a limited number of 1s following the bound pattern.

\myex{false}{ex6-5-bound-pattern}{
    \begin{blockarray}{l}
        n \\
    \begin{block}{l}
       f(n)  \\
    \end{block}
    \end{blockarray}
    \underbrace{
    \begin{blockarray}{c c c c}
        0 & 1 & 2 & 3 \\
    \begin{block}{c c c c}
       0 & 1 & 0 & 0 \\
    \end{block}
    \end{blockarray}
    }_{l}
    \underbrace{
    \begin{blockarray}{c c c}
        4 & 5 & \dots \\
    \begin{block}{c c c}
       1 & 0 & \dots \\
    \end{block}
    \end{blockarray}
    }_{l}
}

So, imagining the string presented above divided in chuncks of lenght \( l \), 
every \( l \) characters there can be a \( f(n)=1 \) for some \( n \) indexed within the chunk.
Since this holds for any size of \( k \), the condition of \( A_l \) 
stating \( \sum_{i=0}^{k} f(i) \leq \frac{k}{l} + 1 \;\; \forall k \in \N \) is satisfied. 
This means that \textbf{at most} 1 of \( (f(0), f(1), \dots, f(l)) \) can be 1, 
for every \( l \) sized chunk in order for all such function to belong to \( A_l \).

The construction of binary sequences denoting results of \( f(i) \; \forall i \in \N \) 
satisfying the above condition are a valid candidate for the injection construction into \( A_l \):
it allows for the function to yield \( 1s \) for the i-th positions with \( i \equiv_l 0 \). 
Such binary construction can be graphically denoted as:

\myex{false}{ex6-5-binary-construction}{
    \underbrace{
    \begin{blockarray}{c c c c}
        0 & 1 & \dots & l-1 \\
    \begin{block}{c c c c}
        \alpha_0 & 0 & \dots & 0 \\
    \end{block}
    \end{blockarray}
    }_{l}
    \underbrace{
    \begin{blockarray}{c c c c}
        l & l+1 & \dots & 2 \cdot l-1 \\
    \begin{block}{c c c c}
        \alpha_1 & 0 & \dots & 0 \\
    \end{block}
    \end{blockarray}
    }_{l}
    \begin{blockarray}{c}
    \begin{block}{c}
       \dots \\
    \end{block}
    \end{blockarray}
    \\\\
    (\alpha_i 0^{l-1})^m \quad \alpha_i \in \{0,1\}, \; \forall i \in \N, \; \forall m \in \N
}
Thus, encoding the possible image of a valid function \( f \in A_l \) along arbitrary 
\( m \cdot l \) sized bit string.
\\

From Theorem 3.23, the set \( \semiinfbinseq \) is uncountable.
Building an injection from \( \semiinfbinseq \) to such above construction enconding \( f \) functions 
in \( A_l \) is proposed.
Let's define the function \( g \):

\myex{false}{ex6-5-g}{
    g: \semiinfbinseq \to A_l, 
    \quad \alpha \in \semiinfbinseq, 
    \quad f \in A_l,
    \quad \alpha_i \in \{0,1\}\\
    \alpha \mapsto f, \quad \forall j \in \N : \; 
    f(j) = 
    \begin{cases}
        \alpha_i, & \text{if } j = l \cdot i \quad i \in \N \\
        0, & \text{otherwise}    
    \end{cases}
}
So we have \( \alpha \) corresponding to a semi infinite binary string, 
and \( \alpha_i \) denoting the i-th bit of the string.
Hence, using the mapping defined, 
\textbf{every possible semi infinite binary sequence has a valid function \( f \) mapping in \( A_l \)}.
\\

To prove the injectivity of the function, 
let \( \alpha, \beta \in \semiinfbinseq \) be two different semi infinite binary sequences.
Let \( i \) be the position where \( \alpha_i \neq \beta_i \).
Let \( g(\alpha) = f_{\alpha}, \; g(\beta) = f_{\beta} \).
\\
By construction of  \( f_{\alpha}, \, f_{\beta}: \;
f_{\alpha}(l \cdot i) = \alpha_i \neq \beta_i = f_{\beta}(l \cdot i)
\;\; \implies \;
f_{\alpha} \neq f_{\beta}\)

\end{document}

\documentclass[unicode,11pt,a4paper,oneside,numbers=endperiod,openany]{scrartcl}

% Required package
\usepackage{amssymb}
\usepackage{graphicx}
\usepackage{amsmath}
\usepackage{matlab-prettifier}
\usepackage{float}
\usepackage[export]{adjustbox}
\usepackage{multirow}
\usepackage{booktabs}
\usepackage{amsthm} % math theorems
\usepackage{ifthen}
\usepackage{physics} % responsive norm, abs, ...
\usepackage{algorithm}
\usepackage{algpseudocode}
\usepackage{array} % for table
\usepackage{mathtools} % for \coloneq
\usepackage{xcolor, soul} % highlight
\usepackage{enumitem} % custom enumeration in lists
\usepackage{dsfont} % 1 for identity matrix
\usepackage{blkarray, bigstrut} % block matrix with labels
\usepackage{tikz} % for drawing

\newcommand\topstrut[1][1.2ex]{\setlength\bigstrutjot{#1}{\bigstrut[t]}}
\newcommand\botstrut[1][0.9ex]{\setlength\bigstrutjot{#1}{\bigstrut[b]}}

\usetikzlibrary{positioning}
\tikzset{
    > = stealth,
    every path/.append style = {
        arrows = -,
    }
}

\renewcommand{\thesubsection}{\arabic{subsection}}
\def\N{\mathbb{N}}
\def\Z{\mathbb{Z}}
\def\Zp{\mathbb{Z}^+}
\def\gcd{\textrm{gcd}}
\def\lcm{\textrm{lcm}}
\def\GF{\textrm{GF}}
\def\deg{\textrm{deg}}
\newcommand{\Zgmult}[1]{\mathbb{Z}_{#1}^{\ast}}
\def\Ring{R}
\def\RingUnits{R^*}


\newtheorem{theorem}{Theorem}[section]
% 1: title, 2: label, 3: content
\newcommand{\myth}[3]{
    \begin{theorem}[#1] 
        \label{#2} 
        #3 
    \end{theorem}
}

% 1: color, 2: content
\newcommand{\mathcolorbox}[2]{\colorbox{#1}{\(\displaystyle #2\)}}

% 1: if numbered equation, 2: label, 3: content
\newcommand{\myex}[3]{
    \ifthenelse{\equal{#1}{true}}{
        \begin{equation} \label{#2} \begin{aligned} #3 \end{aligned} \end{equation}
    }{
        \begin{equation*} \label{#2} \begin{aligned} #3 \end{aligned} \end{equation*}
    }
}

% vector shortcut
\newcommand{\myvec}[1]{\begin{bmatrix} #1 \end{bmatrix}}

% 1: letter of vector, 2: 
\newcommand\vibar[2]{\mathbf{\overline{#1}}_#2}

% 1: caption, 2: label, 3: trim, 4: figure within figures folder
\newcommand{\myfigure}[4]{
    \begin{figure}[htbp]
    \centering
    \caption{#1}
    \label{#2}
    \includegraphics[width=\paperwidth, trim=#3]{./figures/#4}
    \end{figure}
}

% proof step
\newcommand{\pstep}{\overset{.}{\Longrightarrow}}
\newcommand{\spstep}{\overset{.}{\Longleftrightarrow}}

\def\ex2f{f(\mathbf{x}^*)}

\input{assignment.sty}
\begin{document}

\setassignment
\setduedate{Thursday, 28 November 2024, 23:59}

\serieheader
{Discrete Mathematics}
{2024}
{%
\textbf{Student:} Jeferson Morales Mariciano 
\href{mailto:jmorale@ethz.ch}{\(<\)jmorale@ethz.ch\(>\)} \\\\}
{\vspace{-1cm}}%\textbf{Discussed with:} }
{Assignment 10}{}

%----------------------------------------------------------------------------------
\section*{Exercise 10.5, Extension Fields \( (\star) \) \hfill (8 Points)}
Let \( F = \Z_3[x]_{x^3 + 2x^2 + 1} \).

\begin{enumerate}[label=\textbf{\alph*)}]
    \item 
    Prove that \( F \) is a field.

    \item 
    Find a generator of \( F^* \). \textbf{Show your work}.

    \item 
    Find all roots of \( a(y) = y^2 + 2y(x^2 + x) + (x^2 + 2) \) in \( F[y] \).
    \textbf{Show your work}.
\end{enumerate}

%----------------------------------------------------------------------------------
%----------------------------------------------------------------------------------
%----------------------------------------------------------------------------------
\subsection*{a)}

By Definition 5.26, for \( F \) to be a field it must be a non-trivial commutative ring
\( F \) in which every nonzero element is a unit, 
i.e. \( F^* = F \setminus \{ 0 \} \).

\subsubsection*{Properties of \( \Z_3 \) }
Let's check if \( \Z_3 \) is a field.
The short answer positive, given the Theorem 3.23 stating \( \Z_p \) is a field
if and only if \( p \) is prime, which \( 3 \) clearly is. 
This follows from the fact that \( \Zgmult{3} = \Z_3 \setminus \{ 0 \} \) 
is a multiplicative group. 
By Theorem 5.13, 5.15 we know that \( \langle \Zgmult{3}; \odot, \overset{-1}{\ }, 1 \ \rangle \) 
is a group, 
that is is closed under \( \odot \), 
that inherits the associativity of multiplication from the analogous group defined in \( \Z_3 \),
that has a neutral and inverse elements,
and that it is cyclic.
Moreover, its cardinality is \( | \Zgmult{3} | = \varphi(3) = 2 \).

As stated in Example 5.34, using the before-mentioned group we build the ring 
\( \langle \Z_3; \oplus, \ominus, 0, \odot, 1 \rangle \), where we can check that for such algebra:
\( \langle \Z_3; \oplus, \ominus, 0 \rangle \) is an abelian group,
\( \langle \Z_3; \odot, 1 \rangle \) is a monoid,
and that the distributive and commutative law is inherited from the integers,
with its characteristic equal to \( 3 \) (Definition 5.19, Example 5.35) 
and its set of units equal to \( \Zgmult{3} \) (Example 5.39).

As proved at the start, \( \Zgmult{3} \) is a multiplicative group
from the fact that for a ring \( R \), \( R^* \) is a multiplicative group 
since it's the group of units of \( R \), by Lemma 5.18.
Note that it's also an integral domain since every element is a unit besides \( 0 \)
and in a commutative ring any unit is not a zerodivisor, as state in Theorem 5.24,
concluding that a field is an integral domain.

To conclude, we have shown that the algebra 
\( \langle \Zgmult{3}; \odot, \overset{-1}{\ }, 1 \rangle \)
is an abelian group, hence \( F = \Z_3 \) is a field.
The field with \( p \) elements could be denoted as a Galois Field \( \GF(3) \).

\subsubsection*{Properties of \( \Z_3[x] \) }

From Definition 5.25, \( \Z_3[x] \) denotes the set of polynomials \( a(x) \) 
in the indeterminate \( x \)
over the commutative ring and field \( \Z_3 \),
where \( a(x) = \sum_{i=0}^{d} a_i x_i \).
The degree of \( a(x) \) is denoted as \( \deg(a(x)) \), 
which is the greatest \( i \) for which \( a_i \neq 0 \).

Recall that for any commutative ring \( R \),
\( R[x] \) is a commutative ring (Theorem 5.21),
that if \( D \) is an integral domain, then so is \( D[x] \),
and that the units of \( D[x] \) are the constants polynomials that are units of \( D \),
i.e. \( D[x]^* = D^* \) (Lemma 5.22).
So all this properties get into \( \Z_3[x] \), 
denoting the set of polynomials over the field \( \Z_3 \).

\subsubsection*{Properties of \( \Z_3[x]_{x^3 + 2x^2 + 1} \)} 

Let \( m(x) \) be a polynomial of degree \( d \) over \( F \), then
\( F[x]_{m(x)} \overset{\text{def}}{=} \{ a(x) \in F[x] \mid \deg(a(x)) < d \} \)
(Definition 5.34). 
So 
\( \Z_3[x]_{x^3 + 2x^2 + 1} \)
is the set of all polynomials over the field \( \Z_3 \) with degree less than \( 3 \).
Let \( F \) be a finite field with \( q \) elements and let \( m(x) \) be a polynomial 
of degree \( d \) over \( F \). 
Then, \( | F[x]_{m(x)} | = q^d \)
(Lemma 5.34).
Hence, the number of polynomial elements in 
\( \Z_3[x]_{x^3 + 2x^2 + 1} \)
is 
\( | \Z_3[x]_{x^3 + 2x^2 + 1} |
=
| \Z_3 |^{\deg(x^3 + 2x^2 + 1)}
=
3^3 = 27 \).

\( F[x]_{m(x)} \) is a ring with respect to (w.r.t.) addition and multiplication modulo \( m(x) \).
(Lemma 5.35).
Hence, our last step would be to prove that the ring 
\( \Z_3[x]_{x^3 + 2x^2 + 1} \)
is a finite extension field of \( \Z_3 \) with \( 27 \) elements.
To do so, the ring \( F[x]_{m(x)} \) is a field if and only if \( m(x) \) is irreducible
(Theorem 5.37).
I.e. \( \forall a(x) \in F, \; \gcd(a(x), m(x)) = 1 \)
with \( a(x) \neq 0 \; \land \; \deg(a(x)) < deg(m(x)) \),
impliying that \( a(x) \) is invertible in \( F[x]_{m(x)} \).

Let's check if \( m(x) = x^3 + 2x^2 + 1 \) is irreducible w.r.t. \( \Z_3[x] \):
\myex{false}{ex10-5-a-m}{
    m(0) &= 0^3 + 2 \cdot 0^2 + 1 &= 0 + 0 + 1 &= 1 &\equiv_3 1 \\
    m(1) &= 1^3 + 2 \cdot 1^2 + 1 &= 1 + 2 + 1 &= 4 &\equiv_3 1 \\
    m(2) &= 2^3 + 2 \cdot 2^2 + 1 &= 8 + 8 + 1 &= 17 &\equiv_3 2 \\ 
}
Recalling that
a root of \( a(x) \in R[x] \) is any element \( \alpha \in R \) for which 
\( a(\alpha) = 0 \)
(Definition 5.33),
and that for a field \( F \), \( \alpha \in F \) is a root of \( a(x) \) 
if and only if \( x - \alpha \) divides \( a(x) \)
(Lemma 5.29),
we want to find a root of 
\( m(x) = x^3 + 2x^2 + 1 \)
in \( \Z_3[x] \) with elements \( \alpha \in \Z_3 \).
But as we already tried, there are none.
Finally, \( \Z_3[x]_{x^3 + 2x^2 + 1} \) is a field. 

%----------------------------------------------------------------------------------
%----------------------------------------------------------------------------------
%----------------------------------------------------------------------------------
\subsection*{b)}

We know from \textit{a)} that the field \( \Z_3[x]_{x^3 + 2x^2 + 1} \) 
has cardinality 27.
The set of units of the field is 
\( \Z_3^*[x]_{x^3 + 2x^2 + 1} = \Z_3[x]_{x^3 + 2x^2 + 1} \setminus \{ 0 \} \)
with cardinality 26.
Proven also by the fact that the multiplicative group of every finite field \( \GF(q) \) 
is cyclic and
has order \( q-1 \) and \( \varphi(q-1) \) generators (Theorem 5.40).
The factorization of \( 26 = 2 \cdot 13 \), 
and it follows that we have \( \varphi(26) = 12 \) generators.

The order of the elements can be either of its divisors, namely \( 1, 2, 13, 26 \)
(Corollary 5.10).
We need to find an element with order \( 26 \), meaning that generates all the units of the field,
hence the requested generator.

\myex{false}{ex10-5-b}{
    x^1 = x \neq 1 \\
    x^2 = x^2 \neq 1  \\
    x^{13} = x^3 + 2x = 2 \neq 1 
}

A generator of \( F \) is \( x \), since the order of \( x \) is not \( 1, 2, 13 \), 
which left as only option the choice of \( 26 \), meaning generating the whole group.


%----------------------------------------------------------------------------------
%----------------------------------------------------------------------------------
%----------------------------------------------------------------------------------
\subsection*{c)}

To find all roots of \( a(y) = y^2 + 2y(x^2 + x) + (x^2 + 2) \) in \( F[y] \)
we can precompute, as done in the script, some useful polynomials modulo \( m(x) \),
i.e. \( R_{x^3 + 2x^2 + 1} (x^3), R_{x^3 + 2x^2 + 1} (x^4) \), where \( R \) is the 
rest.

\myex{false}{ex10-5-c-rest}{
    &R_{x^3 + 2x^2 + 1} (x^3) = -2x^2 - 1 = x^2 + 2 \\
    &R_{x^3 + 2x^2 + 1} (x^4) = R_{x^3 + 2x^2 + 1} (-2x^3-x) = -2(x^2 + 2) -x 
    = -2 x^2 -4 -x = x^2 + 2x + 2 \\
}

Trivially, any constant \( 0, 1, 2 \) of degree zero does not yield a root. 
We notice that the polynom \( a(y) \) has an unusual structure: 
the term \( (x^2 + 2) \) has no real reason to be in parenthesis but it luckily 
coincide with \( R_{x^3 + 2x^2 + 1} (x^3) \).
Notice that the field \( \Z_3 \) once the coefficient of a term a multiple of \( 3 \),
then it becomes 0 because of modulo \( 3 \).
The term \( 3 x^3 = 3 x^2 + 6 \) is achievable by having \( 2 x^3 \) from injecting 
a polynom into \( a(y) \) to simplify with the already present term.

Rewrite \( a(y) = y^2 + (2 x^2 + 2x) y + (x^2 + 2) \).
The candidate \( x \) seem to be simple good one allowing to simplify:
\myex{false}{ex10-5-x}{
    a(x) = x^2 + (2 x^2 + 2x) \cdot x + x^3 = x^2 + 2 x^3 + 2x^2 + x^3 
    = 3x^2 + 3x^3 = 0
}
Then, \( x \) is a root.\\
Another way to simplify with \( x^3 \) is to increase the degree, i.e. \( x^2 \) as polynom, 
because logically we would still have the same simplification,
let's check the candidate \( x^2 \): 

\myex{false}{ex10-5-xx}{
    a(x^2) = (x^2)^2 + (2 x^2 + 2x) \cdot x^2 + x^3 = x^4 + 2 x^4 + 2x^3 + x^3 
    = 3x^4 + 3 x^3 = 0
}

For a field \( F \), a nonzero polynomial \( a(x) \in F[x] \) of degree \( d \) 
has at most \( d \) roots
(Theorem 5.31).
There are 2 roots, which are \( x, x^2 \), and since \( \deg(m(x)) = 2 \), 
there are no more roots to look for.

\end{document}

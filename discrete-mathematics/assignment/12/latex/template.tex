\documentclass[unicode,11pt,a4paper,oneside,numbers=endperiod,openany]{scrartcl}

% Required package
\usepackage{amssymb}
\usepackage{graphicx}
\usepackage{amsmath}
\usepackage{matlab-prettifier}
\usepackage{float}
\usepackage[export]{adjustbox}
\usepackage{multirow}
\usepackage{booktabs}
\usepackage{amsthm} % math theorems
\usepackage{ifthen}
\usepackage{physics} % responsive norm, abs, ...
\usepackage{algorithm}
\usepackage{algpseudocode}
\usepackage{array} % for table
\usepackage{mathtools} % for \coloneq
\usepackage{xcolor, soul} % highlight
\usepackage{enumitem} % custom enumeration in lists
\usepackage{dsfont} % 1 for identity matrix
\usepackage{blkarray, bigstrut} % block matrix with labels
\usepackage{tikz} % for drawing

\newcommand\topstrut[1][1.2ex]{\setlength\bigstrutjot{#1}{\bigstrut[t]}}
\newcommand\botstrut[1][0.9ex]{\setlength\bigstrutjot{#1}{\bigstrut[b]}}

\usetikzlibrary{positioning}
\tikzset{
    > = stealth,
    every path/.append style = {
        arrows = -,
    }
}

\renewcommand{\thesubsection}{\arabic{subsection}}
\def\N{\mathbb{N}}
\def\Z{\mathbb{Z}}
\def\Zp{\mathbb{Z}^+}
\def\gcd{\textrm{gcd}}
\def\lcm{\textrm{lcm}}
\def\GF{\textrm{GF}}
\def\deg{\textrm{deg}}
\newcommand{\Zgmult}[1]{\mathbb{Z}_{#1}^{\ast}}
\def\Ring{R}
\def\RingUnits{R^*}

% specific assignment 12 
\def\struct{\mathcal{A}}

\newtheorem{theorem}{Theorem}[section]
% 1: title, 2: label, 3: content
\newcommand{\myth}[3]{
    \begin{theorem}[#1] 
        \label{#2} 
        #3 
    \end{theorem}
}

% 1: color, 2: content
\newcommand{\mathcolorbox}[2]{\colorbox{#1}{\(\displaystyle #2\)}}

% 1: if numbered equation, 2: label, 3: content
\newcommand{\myex}[3]{
    \ifthenelse{\equal{#1}{true}}{
        \begin{equation} \label{#2} \begin{aligned} #3 \end{aligned} \end{equation}
    }{
        \begin{equation*} \label{#2} \begin{aligned} #3 \end{aligned} \end{equation*}
    }
}

% vector shortcut
\newcommand{\myvec}[1]{\begin{bmatrix} #1 \end{bmatrix}}

% 1: letter of vector, 2: 
\newcommand\vibar[2]{\mathbf{\overline{#1}}_#2}

% 1: caption, 2: label, 3: trim, 4: figure within figures folder
\newcommand{\myfigure}[4]{
    \begin{figure}[htbp]
    \centering
    \caption{#1}
    \label{#2}
    \includegraphics[width=\paperwidth, trim=#3]{./figures/#4}
    \end{figure}
}

% proof step
\newcommand{\pstep}{\overset{.}{\Longrightarrow}}
\newcommand{\spstep}{\overset{.}{\Longleftrightarrow}}

\def\ex2f{f(\mathbf{x}^*)}

\input{assignment.sty}
\begin{document}

\setassignment
\setduedate{Thursday, 12 December 2024, 23:59}

\serieheader
{Discrete Mathematics}
{2024}
{%
\textbf{Student:} Jeferson Morales Mariciano 
\href{mailto:jmorale@ethz.ch}{\(<\)jmorale@ethz.ch\(>\)} \\\\}
{\vspace{-1cm}}%\textbf{Discussed with:} }
{Assignment 12}{}

%----------------------------------------------------------------------------------
\section*{Exercise 12.6, Semantics of Predicate Logic \hfill (8 Points)}
\textbf{Prove} the following statements using the semantics of predicate logic 
(Definition 6.36).
Do not use other results from the lecture notes.

\begin{enumerate}[label=\textbf{\alph*)}]
    \item 
    \( \exists x \, \forall y \, P(x,y) \models \exists x \, P(x,f(x)) \).

    \item 
    \( \neg (\forall x \; P(x)) \models \exists x \, \neg P(x) \).

\end{enumerate}

%----------------------------------------------------------------------------------
%----------------------------------------------------------------------------------
%----------------------------------------------------------------------------------
\subsection*{a)}

Let \( \struct = (U, \phi, \psi, \xi) \) be an arbitrary suitable structure for the formulas.
Assume \( \struct \) is a model for the first formula, 
i.e. \( \struct \models \exists x \, \forall y \, P(x,y) \).
There are no free variables in the formula since all are bound, hence it is a closed formula. 
The \( P^\struct \) is omitted for simplicity, as it remains a free predicate symbol 
not involved in the proof.
% For structure \( \struct \) to be suitable, defining the free symbols \( P, f \) of both formulas 
% is required (Definition 6.35). \\
By Definition 6.9 of suitable interpretation for a formula:

\[
\struct \models \exists x \, \forall y \, P(x,y) 
\implies 
\struct\left( 
    \exists x \, \forall y \, P(x,y) 
\right) = 1
\]

\noindent Recall the semantics of \( \exists \), 
\( \struct(\exists x \, G) = 1 \) if \( \struct_{[x \to u]}(G) = 1 \) for some \( u \in U \)
(Definition 6.36).
Since the formula \( \forall y \, P(u,y) \) is true for some \( u \in U^\struct \), 
applying the semantics results in overriding \( \xi(x) = u \) for:

\[
\struct_{[x \to u]} 
\left(
    \forall y \, P(x,y)
\right) = 1 \quad \text{for some } u \in U^\struct
\]

\noindent Recall the semantics of \( \forall \),
\( \struct(\forall x \, G) = 1 \) if \( \struct_{[x \to u]}(G) = 1 \) for all \( u \in U \)
(Definition 6.36).
Since the formula \( P(u,w) \) is true (for some \( u \in U^\struct \) and) for all \( w \in U^\struct \),
applying the semantics results in overriding also \( \xi(y) = w \) for:

\[
\struct_{[x \to u][y \to w]} \left(
    P(x,y)
\right) = 1 
\quad \text{for some } u \in U^\struct
\text{ and for all } w \in U^\struct 
\]

\noindent Notice that the contraint on \( w \) is very loose,
and most importantly, that it is satisfied by definition of \( \phi : U^k \to U \) 
with \( k = 1 \) for any arbitrary unary function symbol \( f \) in the structure
(Definition 6.36).
Specifically, \( \phi : U \to U \) 
where \( w = \phi(f)(\struct(x)) = \phi(f)(\xi(x)) = \phi(f)(u) \) 
with \( u \in U^\struct \) and \( f \) arbitrary. 
Instead of \( \phi(f) \), the notation \( f^\struct \) is used for simplicity.
Thus:

\[
\struct_{[x \to u][y \to f^\struct (x)]} \left(
    P(x,y)
\right) = 1 
\quad \text{for some } u \in U^\struct
\text{ and arbitrary } f^\struct
\]

\noindent By using the recursivity of the semantics of Predicate Logic,
the unary function symbols is embedded as a term in the formula
(Definition 6.36). 
The superscript \( \struct \) is omitted for simplicity on unary function symbol \( f \) of 
arbitrary suitable structure \( \struct \).

\[
\struct_{[x \to u]} \left(
    P(x, f(x))
\right) = 1 
\quad \text{for some } u \in U^\struct
\]

\noindent Thus, by applying the semantics of \( \exists \) again,
since the formula \( P(u, f(u)) \) is true for some \( u \in U^\struct \),
the proof to show that the structure \( \struct \) is a model for the second formula is completed:

\[
\struct (\exists x \, P(x, f(x))) = 1 
\]

\noindent Finally,

\[
\exists x \, \forall y \, P(x,y) \models \exists x \, P(x,f(x))
\]

%----------------------------------------------------------------------------------
%----------------------------------------------------------------------------------
%----------------------------------------------------------------------------------
\subsection*{b)}

Let \( \struct = (U, \phi, \psi, \xi) \) be an arbitrary suitable structure for the formulas.
Assume \( \struct \) is a model for the first formula, 
i.e. \( \struct \models \neg (\forall x \, P(x)) \).
There are no free variables in the formula since all are bound, hence it is a closed formula. 
The \( P^\struct \) is omitted for simplicity, as it remains a free predicate symbol 
not involved in the proof.
By Definition 6.9 of suitable interpretation for a formula:

\[
\struct \models \neg (\forall x \, P(x)) 
\implies 
\struct\left( 
     \neg (\forall x \, P(x))
\right) = 1
\]

\noindent Recall the semantics of \( \neg \) for a formula in predicate logic
(Definition 6.16, 6.24).
Since \( \forall x \, P(x) \) is a formula in predicate logic (Definition 6.36), 
then also its negation is (Definition 6.31). 
Since \( \struct(\neg F) = 1 \iff \struct(F) = 0 \), recursively decoupling the formula yields:

\[
\struct\left( 
    \forall x \, P(x)
\right) = 0
\]

\noindent Recall the semantics of \( \forall \),
\( \struct(\forall x \, G) = 1 \) if \( \struct_{[x \to u]}(G) = 1 \) for all \( u \in U \),
otherwise \( \struct(\forall x \, G) = 0 \) 
(Definition 6.36).
Since the formula \( P(u) \) is not true for all \( u \in U^\struct \),
applying the semantics results in overriding \( \xi(x) = u \) for:

\[
\struct_{[x \to u]}\left( 
    P(x)
\right) = 0
\quad
\text{ for all } u \in U^\struct
\]

\noindent Reasoning about the formula \( P(u) \) not being true for all \( u \in U^\struct \),
means that there is some, at least one, \( P(u) \) that is false.
Rephrasing the statement, \( P(u) \) is false for some \( u \in U^\struct \),
or equivalently, \( \neg P(u) \) is true for some \( u \in U^\struct \) 
by semantics of \( \neg \)
(Definition 6.16).
Thus:

\[
\struct_{[x \to u]}\left( 
    \neg P(x)
\right) = 1
\quad
\text{ for some } u \in U^\struct
\]

\noindent Recall semantics of \( \exists \),
\( \struct(\exists x \, G) = 1 \) if \( \struct_{[x \to u]}(G) = 1 \) for some \( u \in U \)
(Definition 6.36).
Since the formula \( \neg P(u) \) is true for some \( u \in U^\struct \),
applying the semantics results in:

\[
\struct\left( 
    \exists x \, \neg P(x)
\right) = 1
\]

\noindent Finally, this complete the proof that 
for any arbitrary structure for the first formula, it is also a model for the second formula:

\[
\neg \left( \forall x \, P(x) \right) \models \exists x \, \neg P(x)
\]

\end{document}

\documentclass[unicode,11pt,a4paper,oneside,numbers=endperiod,openany]{scrartcl}

% Required package
\usepackage{amssymb}
\usepackage{graphicx}
\usepackage{amsmath}
\usepackage{matlab-prettifier}
\usepackage{float}
\usepackage[export]{adjustbox}
\usepackage{multirow}
\usepackage{booktabs}
\usepackage{amsthm} % math theorems
\usepackage{ifthen}
\usepackage{physics} % responsive norm, abs, ...
\usepackage{algorithm}
\usepackage{algpseudocode}
\usepackage{array} % for table
\usepackage{mathtools} % for \coloneq
\usepackage{xcolor, soul} % highlight
\usepackage{enumitem} % custom enumeration in lists
\usepackage{dsfont} % 1 for identity matrix
\usepackage{blkarray, bigstrut} % block matrix with labels
\usepackage{tikz} % for drawing

\newcommand\topstrut[1][1.2ex]{\setlength\bigstrutjot{#1}{\bigstrut[t]}}
\newcommand\botstrut[1][0.9ex]{\setlength\bigstrutjot{#1}{\bigstrut[b]}}

\usetikzlibrary{positioning}
\tikzset{
    > = stealth,
    every path/.append style = {
        arrows = -,
    }
}

\renewcommand{\thesubsection}{\arabic{subsection}}
\def\N{\mathbb{N}}
\def\Z{\mathbb{Z}}
\def\Zp{\mathbb{Z}^+}
\def\gcd{\textrm{gcd}}
\def\lcm{\textrm{lcm}}
\def\GF{\textrm{GF}}
\def\deg{\textrm{deg}}
\newcommand{\Zgmult}[1]{\mathbb{Z}_{#1}^{\ast}}
\def\Ring{R}
\def\RingUnits{R^*}

% specific assignment 11 
\def\myp{\mathcal{P}}
\def\mys{\mathcal{S}}

\newtheorem{theorem}{Theorem}[section]
% 1: title, 2: label, 3: content
\newcommand{\myth}[3]{
    \begin{theorem}[#1] 
        \label{#2} 
        #3 
    \end{theorem}
}

% 1: color, 2: content
\newcommand{\mathcolorbox}[2]{\colorbox{#1}{\(\displaystyle #2\)}}

% 1: if numbered equation, 2: label, 3: content
\newcommand{\myex}[3]{
    \ifthenelse{\equal{#1}{true}}{
        \begin{equation} \label{#2} \begin{aligned} #3 \end{aligned} \end{equation}
    }{
        \begin{equation*} \label{#2} \begin{aligned} #3 \end{aligned} \end{equation*}
    }
}

% vector shortcut
\newcommand{\myvec}[1]{\begin{bmatrix} #1 \end{bmatrix}}

% 1: letter of vector, 2: 
\newcommand\vibar[2]{\mathbf{\overline{#1}}_#2}

% 1: caption, 2: label, 3: trim, 4: figure within figures folder
\newcommand{\myfigure}[4]{
    \begin{figure}[htbp]
    \centering
    \caption{#1}
    \label{#2}
    \includegraphics[width=\paperwidth, trim=#3]{./figures/#4}
    \end{figure}
}

% proof step
\newcommand{\pstep}{\overset{.}{\Longrightarrow}}
\newcommand{\spstep}{\overset{.}{\Longleftrightarrow}}

\def\ex2f{f(\mathbf{x}^*)}

\input{assignment.sty}
\begin{document}

\setassignment
\setduedate{Thursday, 5 December 2024, 23:59}

\serieheader
{Discrete Mathematics}
{2024}
{%
\textbf{Student:} Jeferson Morales Mariciano 
\href{mailto:jmorale@ethz.ch}{\(<\)jmorale@ethz.ch\(>\)} \\\\}
{\vspace{-1cm}}%\textbf{Discussed with:} }
{Assignment 11}{}

%----------------------------------------------------------------------------------
\section*{Exercise 11.4, Combining Proof Systems \( (\star) \) \hfill (8 Points)}
Let 
\[
\Sigma = (\mathcal{S}, \mathcal{P}, \tau, \phi)
\]
be a complete and sound proof system.

\begin{enumerate}[label=\textbf{\alph*)}]
    \item 
    Define \( \mathcal{P'} \) and \( \mathcal{\phi'} \) so that 
    \[
    \Sigma' = (\mathcal{S} \times \mathcal{S} \times \mathcal{S}, 
        \mathcal{P'}, \tau', \phi')
    \]
    is a complete and sound proof system (and prove it!), where
    \[
    \mathcal{\tau'}((s_1, s_2, s_3)) = 1 
    \iff 
    \text{ at least } 2 \text{ among } \tau(s_1), \tau(s_2), \tau(s_3) 
    \text{ are equal to } 1
    \]

    \item 
    Let 
    \[
    \overline{\Sigma} = (\mathcal{S}^2, 
        \overline{\mathcal{P}}, \overline{\tau}, \overline{\phi})
    \]
    be a complete and sound proof system with
    \myex{true}{ex11-4-b-cond}{
    \overline{\tau}((s_1, s_2)) = 1 
    \iff &
    \text{exactly } 1 \text{ of the statements is true in } \Sigma,\\
    & \text{ that is, } \tau(s_1) = 1 \text{ or } \tau(s_2) = 1, 
        \text{but not both.}
    }
    Define \( \mathcal{P^*} \) and \( \mathcal{\phi^*} \) so that
    \( \Sigma^* = (\mathcal{S}, \mathcal{P^*}, \tau^*, \phi^*) \)
    is a complete and sound proof system (and prove it!), 
    where
    \[
    \tau^*(s) = 1 \iff \tau(s) = 0
    \]

\end{enumerate}

%----------------------------------------------------------------------------------
%----------------------------------------------------------------------------------
%----------------------------------------------------------------------------------
\subsection*{a)}

Remark \( \mys' = \mys \times \mys \times \mys 
    = \{ (s_1, s_2, s_3) \mid \forall i \in \{ 1, 2, 3 \} \; s_i \in \mys \} \), 
    where \( | \mys' | = | \mys |^3 \). \\
Let 
\myex{false}{ex11-4-a-defs}{ 
    &
    \myp' = \myp \times \myp \times \myp
        = \{ (p_1, p_2, p_3) \mid \forall i \in \{ 1, 2, 3 \}  \; p_i \in \myp  \}  \\
    &
    \phi' : \mys' \times \myp' \to \{ 0, 1 \}, \quad  
    s' = (s_1, s_2, s_3) \in \mys', \; p' = (p_1, p_2, p_3) \in \myp' \\
    &
    (s', p')  \mapsto
    \begin{cases}
        1 & \text{ if } \forall i \forall j \in \{ 1, 2, 3 \} 
            \; \phi(s_i, p_i) = 1 \; \land \; \phi(s_j, p_j) = 1 \; \land \; i \neq j \\
        0 & \text{ otherwise}
    \end{cases}
}

\noindent\textbf{Completeness}: \\
\noindent Assume \( \tau'(s') = 1 \). 
Then, at least \( 2 \) of \( \tau(s_1), \tau(s_2), \tau(s_3) \) 
must equal \( 1 \) by definition of \( \tau' \) in \textbf{a)}.  \\
By completeness of proof system 
\( \Sigma\), every true statement has a proof, 
i.e. \( \forall i \in \{ 1, 2, 3 \}, \; \forall s_i \in \mys, 
    \; \tau(s_i) = 1 \implies \exists p_i \in \myp, \; \phi(s_i, p_i) = 1 \). \\
Since \( 2 \) among \( s_1, s_2, s_3 \) satisfy \( \tau(s_i) = 1  \; \land \; \tau(s_j) = 1 \)
by definition of \( \phi' \) 
(assuming the exercise required distinct elements, thus \( i \neq j \), 
which is a stronger case than repeated ones),
it is possible to contruct the proof to verify correctness of at least 2 of such statements,
i.e. \( \forall s' \in \mys', \; s' = (s_1, s_2, s_3), \; \tau'(s') = 1 \implies 
    \; \exists p' \in \myp', \; p' = (p_1, p_2, p_3) \; \land \; \phi'(s', p') = 1 \),
where \( 2 \) statements are verified by the before mentioned completeness of \( \Sigma \).
Moreover, there are at least as many proofs as true statements since 
for each true statement there exist at least a proof.
The definition of the proof system \( \Sigma' \) ensures it has at least \( 2 \) truth statements.
When 2 true statements with their proofs and 1 false is proposed, then the third proof is just
one of the already present, as the distinct elements required are only 2, 
the verification function will just yield false.  \\
Thus, \( \Sigma' \) is complete. \newline

\noindent \textbf{Soundness}:
\noindent Assume \( \phi'(s', p') = 1 \).
Then, \( \forall i \forall j \in \{ 1, 2, 3 \} \; \phi(s_i, p_i) = 1 \; \land \; \phi(s_j, p_j) = 1 
    \; \land \; i \neq j \) by defintion of \( \phi' \). \\
By soundness of proof system \( \Sigma \), 
no false statement has a valid proof proving it true,
or equivalently, if a statement does not have a proof, it must be false,
i.e. \( \forall i \in \{ 1, 2, 3 \}, \; \forall p_i \in \myp, \; \forall s_i \in \mys, 
    \; \phi(s_i, p_i) = 1 \implies \tau(s_i) = 1 \). \\
Hence, no proof for false statement exists, specifically satisfying \( \tau' \) definition 
of having at least \( 2 \) of \( \tau(s_i) = 1 \). \\
Thus, \( \Sigma' \) is sound.


%----------------------------------------------------------------------------------
%----------------------------------------------------------------------------------
%----------------------------------------------------------------------------------
\subsection*{b)}

Let 
\myex{false}{ex11-4-b-defs}{ 
    &
    \myp^* = \myp \times \myp
        = \{ (p_1, p_2) \mid \forall i \in \{ 1, 2 \}  \; p_i \in \myp  \}  \\
    &
    \phi' : \mys \times \myp^* \to \{ 0, 1 \}, \quad  
    s \in \mys', \; p^* = \overline{p} = (p_1, p_2) \in \myp^* \\
    &
    (s, p^*)  \mapsto
    \begin{cases}
        1 & \text{ if } 
            \left( 
            \exists t \in \mys, \; \exists \overline{p} \in \overline{\myp}, 
            \; \overline{\phi}((s, t), \overline{p}) = 1 
            \right)
            \; \land \; 
            \left( \exists i \in \{ 1, 2 \}, \; \phi(t, \overline{p_i}) = 1 \right) \\
        0 & \text{ otherwise}
    \end{cases}
}

Clarifying that the definition of the complete proof system \( \overline{\Sigma} \)
ensures that there is at least 1 true statement in order to use its \( \overline{\tau} \),
\( t \in \mys \) wants to be such a \textbf{true} statement we use to distinguish
our statement \( s \) from the proof system \( \Sigma^* \) in order to be able to verify,
together with \( \overline{\phi} \), 
if our statement w.r.t. its truth function is either true or false.
Such true statement is found in polynomial time by a linear scan among the statements and proofs
and checking with \( \phi(t, p) \).
Otherwise, a tautology can be used as such in the set of rules admit so.
Recall there are at least as many proofs as true statements from completeness of \( \Sigma \).

\noindent \textbf{Completeness}:
\noindent Assume  \( \tau^*(s) = 1 \). 
Then, \( \tau(s) = 0 \) by definition of \( \tau^* \). \\
By soundness of \( \Sigma \), no false statements have a proof.
Using the clarified true statement \( t \) by checking \( \phi(t, p) = 1 \),
we use it within \( \overline{\phi}((s, t), \overline{p}) = 1 \),
By soundness of \( \overline{\Sigma} \), \( \forall p \forall s, \; 
    \overline{\phi}((s, t), \overline{p}) = 1 \implies \overline{\tau}((s, t)) = 1 \).
By definition of \( \overline{\tau} \), only one of \( s, t \) is true,
and since we known from assumption premises that \( s \) is false because \( \tau(s) = 0 \),
then \( t \) must be true. \\
Hence, we can construct \( \overline{p} = p^* \) for \( \phi^*(s, p^*) = 1 \) 
because \( p^* \) for sure is not a proof for \( s \). 
Thus, \( \Sigma^* \) is complete. \newline

\noindent \textbf{Soundness}:
\noindent Assuming \( \phi^*(s, p^*) = 1 \). 
Then, \( \overline{\phi}((s, t), \overline{p}) = 1 \; \land \; \phi(t, \overline{p_i}) = 1 \) for 
some \( i \). \\
By soundness of \( \Sigma \), \(\phi(t, \overline{p_i}) = 1 \implies \tau(t) = 1 \). \\
By soundness of \( \overline{\Sigma} \), \( \overline{\phi}((s, t), \overline{p}) = 1 \) 
and \( \tau(t) = 1 \) implies \( \tau(s) = 0 \). \\
Hence, by definition of \( \tau^* \), \( \tau(s) = 0 \iff \tau^*(s) = 1 \).
Thus, \( \Sigma^* \) is sound.

\end{document}

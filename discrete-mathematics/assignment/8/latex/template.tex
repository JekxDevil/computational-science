\documentclass[unicode,11pt,a4paper,oneside,numbers=endperiod,openany]{scrartcl}

% Required package
\usepackage{amssymb}
\usepackage{graphicx}
\usepackage{amsmath}
\usepackage{matlab-prettifier}
\usepackage{float}
\usepackage[export]{adjustbox}
\usepackage{multirow}
\usepackage{booktabs}
\usepackage{amsthm} % math theorems
\usepackage{ifthen}
\usepackage{physics} % responsive norm, abs, ...
\usepackage{algorithm}
\usepackage{algpseudocode}
\usepackage{array} % for table
\usepackage{mathtools} % for \coloneq
\usepackage{xcolor, soul} % highlight
\usepackage{enumitem} % custom enumeration in lists
\usepackage{dsfont} % 1 for identity matrix
\usepackage{blkarray, bigstrut} % block matrix with labels
\usepackage{tikz} % for drawing

\newcommand\topstrut[1][1.2ex]{\setlength\bigstrutjot{#1}{\bigstrut[t]}}
\newcommand\botstrut[1][0.9ex]{\setlength\bigstrutjot{#1}{\bigstrut[b]}}

\usetikzlibrary{positioning}
\tikzset{
    > = stealth,
    every path/.append style = {
        arrows = -,
    }
}

\renewcommand{\thesubsection}{\arabic{subsection}}
\def\N{\mathbb{N}}
\def\Z{\mathbb{Z}}
\def\Zp{\mathbb{Z}^+}
\def\gcd{\textrm{gcd}}
\def\lcm{\textrm{lcm}}
\newcommand{\Zgmult}[1]{\mathbb{Z}_{#1}^{\ast}}


\newtheorem{theorem}{Theorem}[section]
% 1: title, 2: label, 3: content
\newcommand{\myth}[3]{
    \begin{theorem}[#1] 
        \label{#2} 
        #3 
    \end{theorem}
}

% 1: color, 2: content
\newcommand{\mathcolorbox}[2]{\colorbox{#1}{\(\displaystyle #2\)}}

% 1: if numbered equation, 2: label, 3: content
\newcommand{\myex}[3]{
    \ifthenelse{\equal{#1}{true}}{
        \begin{equation} \label{#2} \begin{aligned} #3 \end{aligned} \end{equation}
    }{
        \begin{equation*} \label{#2} \begin{aligned} #3 \end{aligned} \end{equation*}
    }
}

% vector shortcut
\newcommand{\myvec}[1]{\begin{bmatrix} #1 \end{bmatrix}}

% 1: letter of vector, 2: 
\newcommand\vibar[2]{\mathbf{\overline{#1}}_#2}

% 1: caption, 2: label, 3: trim, 4: figure within figures folder
\newcommand{\myfigure}[4]{
    \begin{figure}[htbp]
    \centering
    \caption{#1}
    \label{#2}
    \includegraphics[width=\paperwidth, trim=#3]{./figures/#4}
    \end{figure}
}

% proof step
\newcommand{\pstep}{\overset{.}{\Longrightarrow}}

\def\ex2f{f(\mathbf{x}^*)}

\input{assignment.sty}
\begin{document}

\setassignment
\setduedate{Thursday, 14 November 2024, 23:59}

\serieheader
{Discrete Mathematics}
{2024}
{%
\textbf{Student:} Jeferson Morales Mariciano 
\href{mailto:jmorale@ethz.ch}{\(<\)jmorale@ethz.ch\(>\)} \\\\}
{\vspace{-1cm}}%\textbf{Discussed with:} }
{Assignment 8}{}

%----------------------------------------------------------------------------------
\section*{Exercise 8.5, Inner Direct Products \( (\star) \) \hfill (8 Points)}

\begin{enumerate}[label=\textbf{\alph*)}]
    \item 
    Let \( \langle G; \ast, \widehat{\ }, e \rangle \) be a commutative group.
    Let \( H \) and \( K \) be subgroups of \( G \) such that

    \begin{enumerate}[label=(\roman*)]
        \item \( G = \{ h \ast k \mid h \in H, k \in K \} \),
        \item \( H \cap K = \{ e \} \).
    \end{enumerate}
    
    Prove that \( G \) is isomorphic to the direct product \( H \times K \).
    In this case, \( G \) is called the \textit{inner} direct product of \( H \) and \( K \).

    \item 
    Use the previous subtask to prove that 
    \( \langle \Z^{\ast}_{15}, \odot_{15} \rangle \simeq \Z_2 \times \Z_4 \).
    You can use the subtask even if you have not proved it.
    \textbf{Do not} prove the isomorphism directly.
\end{enumerate}

\subsection*{a)}

By the exercise statement, all the elements of \( G \) can be written as 
\( h \ast k \) for some \( h \in H \) and \( k \in K \) 
by the first condition \( (i) \),
thus all the elements of \( H \) and \( K \) are in \( G \) as well because 
\( \forall \, h \in H, \, h \ast e \in G \) and \( \forall \, k \in K, \, e \ast k \in G \).
The group \( G \) is stated to be abelian, 
so \( H \) and \( K \) are abelian as well 
and recall \( \ast \) is associative by Definition 5.7 of Group.

For the subgroups \( \langle H; \ast, \widehat{\ }, e \rangle \) 
and \( \langle K; \ast, \widehat{\ }, e \rangle \) of \( G \),
their direct product \( H \times K \) is defined as algebra 
\( \langle H \times K; \star, \widehat{\ }, e \rangle \).
where the component wise operation \( \star \) is defined as

\myex{false}{ex8-5-star}{
( a_1, b_1 ) \, \star \, ( a_2, b_2 ) =  ( a_1 \ast a_2, \, b_1 \ast b_2 )
\quad a_1, a_2 \in H, \, b_1, b_2 \in K
\qquad &\text{(Definition 5.9)} \\
\pstep \;
( h_1, k_1 ) \, \star \, ( h_2, k_2 ) =  ( h_1 \ast h_2, \, k_1 \ast k_2 )
\quad \, (h_1, k_1), (h_2, k_2) \in H \times K &
}.

From Lemma 5.4 \( \langle G_1 \times \dots \times G_n; \ast \rangle \) is a group where the neutral element 
and the inversion operation are component wise in the respective groups, but by Definition 5.11 of Subgroups,
\( H, K \) and group \( G \) share the same neutral element \( e \), inverse operation \( \widehat{\ } \), 
and are closed under the operation \( \ast \).

Let's define the mapping function \( \psi \).
\myex{false}{ex8-5-psi}{
    \psi : G \to H \times K \qquad 
    h \ast k \mapsto (h, k)
    \quad \forall \, h \ast k \in G
}

The group homomorphism property of \( \psi \) is check as follows
\myex{false}{ex8-5-homo}{
&\psi(a \ast b) 
    = \psi(a) \, \star \, \psi(b) 
    \quad \forall \, a, b \in G 
& \hspace{1cm} \hfill \text{(CHECK Definition 5.10)}
\\
&\psi((h_1 \ast k_1) \, \ast \, (h_2 \ast k_2) ) 
    = \psi(h_1 \ast k_1) \, \star \, \psi(h_2 \ast k_2) 
    = ( (h_1, h_2), \, (k_1, k_2) ) 
& \hspace{1cm} \hfill \text{(group elems notation)}
\\
&\psi((h_1 \ast k_1) \, \ast \, (h_2 \ast k_2) ) 
\; \pstep \; 
\psi(h_1 \ast (k_1 \, \ast \, h_2) \ast k_2 )  
& \hspace{1cm} \hfill \text{(associativity)}
\\
& \; \pstep \; 
\psi(h_1 \ast (h_2 \, \ast \, k_1) \ast k_2 )  
& \hspace{1cm} \hfill \text{(commutativity)}
\\
& \; \pstep \; 
\psi((h_1 \ast h_2) \, \ast \, (k_1 \ast k_2) )  
& \hspace{1cm} \hfill \text{(associativity)}
\\
& \; \pstep \; 
\psi(h \, \ast \, k )  
& \hspace{1cm} \hfill h = h_1 \ast h_2, \; k = k_1 \ast k_2
\\
& \; \pstep \; 
(h, k)
\; \pstep \; 
( (h_1 \ast h_2), \, (k_1 \ast k_2) )
\; \pstep \; 
( (h_1, k_1) \, \star \, (h_2, k_2) )
\\ 
& \; \pstep \; 
\psi(h_1 \ast k_1) \, \star \, \psi(h_2 \ast k_2)
\; \pstep \; 
\psi(a) \, \star \, \psi(b)
& \hspace{1cm} \hfill \qed
}

From \( G \) to direct product of 2 subgroups of \( G \).\\

Injectivity: let \( \psi(a) = \psi(b) \) for some \( a, b \in G \), then
\myex{false}{ex8-5-inj}{
    \psi(a) = \psi(b) 
    \; \pstep \; 
    (h_a, k_a) = (h_b, k_b) 
    \; \pstep \; 
    h_a = h_b, \, k_a = k_b
    \\
    \; \pstep \; 
    a = b
    \; \pstep \; 
    \psi(a) = \psi(b)
    \; \pstep \; 
    \psi \text{ is injective}
}

For any \( (h, k) \in H \times K \), it exists an element \( a \) in \( G \) 
by definition \( (i) \) such that \( a = h \ast k \) and \( \psi(a) = (h, k) \).
Thus, \( \psi \) is surjective.
Finally, \( psi \) is bijective, then the homomorphism \( \psi \) is an isomorphism
and \( G \simeq H \times K \),
where \( G \) is the inner direct product of \( H \) and \( K \).


%----------------------------------------------------------------------------------
%----------------------------------------------------------------------------------
%----------------------------------------------------------------------------------
\subsection*{b)}

The group \( \Z^{\ast}_{15} \) is the set of all integers \( a \) 
such that \( a \in Z_{15} \, \land \, \gcd(a, 15) = 1 \) by Definition 5.16.
Hence, the group \( \Z^{\ast}_{15} = \{ 1, 2, 4, 7, 8, 11, 13, 14 \} \) and
\( | \Zgmult{15} | = 8 \) which is the order of \( \Zgmult{15} \) due to Definition 5.13.
By Definition 5.15, such group is not cyclic.

Then, the group \( \Z_2 \times \Z_4 \) is closed under modular addition \( \oplus \) 
because for modular multiplication \( \odot \) 
in general the group \( \Z_n \) may have elements without the multiplication inverse, as 
\( \Z_{12} \) with the element \( 8 \) which has no inverse. 
Such group \( \Z_2 \times \Z_4 \) can be seen as the direct product of the groups \( \Z_2 \) and \( \Z_4 \) 
from the previous subtask.
From Theorem 5.7, \( \langle \Z_n; \oplus \rangle \) is the standard notation for the cyclic group of order \( n \).
The definition is \( \Z_2 \times \Z_4  = \{ (0, 0), (0, 1), (0, 2), (0, 3), (1, 0), (1, 1), (1, 2), (1, 3) \} \).
So we know that \( \Z_2, \Z_4 \) are cyclic groups.

Notice \( | \Z_2 \times \Z_4 | = 8 = | \Z^{\ast}_{15} | \)
so a totally well defined mapping \( \psi \) could be defined as in the previous subtask,
and it can be shown that it is both injective and surjective, hence bijective and an isomorphism.

The order of elements from \( \Z^{\ast}_{15} \) are as follows:
\( ord = 2 \) for \( 4, 11, 14 \),
\( ord = 4 \) for \( 2, 7, 8, 13 \),
and \( ord = 1 \) for \( 1 \) as it is the multiplicative inverse and also self inverse.

The cyclic subgroups of \( \Zgmult{15} \) generated by generator elements \( 2, 4 \) 
are \( \langle 2 \rangle = \{ 1, 2, 4, 8 \} \) with \( ord( \langle 2 \rangle ) = 4\) 
and \( \langle 4 \rangle = \{ 1, 4 \} \) with \( ord( \langle 4 \rangle ) = 2 \) respectively
using Definition 5.14.

By Theorem 5.7, a cyclic group of order \( n \) is isomorphic to \( \langle \Z_n ; \oplus \rangle \) 
and abelian.
So \( \langle 2 \rangle, \langle 4 \rangle \) are isomorphic to \( \Z_4, \Z_2 \) w.r.t. their order,
i.e. \( \langle 2 \rangle \simeq \langle \Z_4; \oplus \rangle 
\,, \, \langle 4 \rangle \simeq \langle \Z_2 ; \oplus \rangle \). 

Hence, from the previous subtask \textbf{a)},
the group \( G \) can be seen as \( \Zgmult{15} \) and the direct product of \( H \times K \)
would be the direct product of the cyclic subgroups
\( \langle 2 \rangle, \langle 4 \rangle \), which are isomorphic to \( \Z_2 \times \Z_4 \),
i.e. \( \Zgmult{15} \, \simeq \, \langle 2 \rangle \times \langle 4 \rangle \, \simeq \, \Z_2 \times \Z_4 \).
The proof is complete and a direct isomorphism is not needed from the exercise statement.



\end{document}

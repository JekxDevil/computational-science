\documentclass[unicode,11pt,a4paper,oneside,numbers=endperiod,openany]{scrartcl}

% Required package
\usepackage{amssymb}
\usepackage{graphicx}
\usepackage{amsmath}
\usepackage{matlab-prettifier}
\usepackage{float}
\usepackage[export]{adjustbox}
\usepackage{multirow}
\usepackage{booktabs}
\usepackage{amsthm} % math theorems
\usepackage{ifthen}
\usepackage{physics} % responsive norm, abs, ...
\usepackage{algorithm}
\usepackage{algpseudocode}
\usepackage{array} % for table
\usepackage{mathtools} % for \coloneq
\usepackage{xcolor, soul} % highlight
\usepackage{enumitem} % custom enumeration in lists

\renewcommand{\thesubsection}{\arabic{subsection}}

\newtheorem{theorem}{Theorem}[section]
% 1: title, 2: label, 3: content
\newcommand{\myth}[3]{
    \begin{theorem}[#1] 
        \label{#2} 
        #3 
    \end{theorem}
}

% 1: color, 2: content
\newcommand{\mathcolorbox}[2]{\colorbox{#1}{\(\displaystyle #2\)}}

% 1: if numbered equation, 2: label, 3: content
\newcommand{\myex}[3]{
    \ifthenelse{\equal{#1}{true}}{
        \begin{equation} \label{#2} \begin{aligned} #3 \end{aligned} \end{equation}
    }{
        \begin{equation*} \label{#2} \begin{aligned} #3 \end{aligned} \end{equation*}
    }
}

% vector shortcut
\newcommand{\myvec}[1]{\begin{bmatrix} #1 \end{bmatrix}}

% 1: letter of vector, 2: 
\newcommand\vibar[2]{\mathbf{\overline{#1}}_#2}

% 1: caption, 2: label, 3: trim, 4: figure within figures folder
\newcommand{\myfigure}[4]{
    \begin{figure}[htbp]
    \centering
    \caption{#1}
    \label{#2}
    \includegraphics[width=\paperwidth, trim=#3]{./figures/#4}
    \end{figure}
}

% proof step
\newcommand{\pstep}{\overset{.}{\Longrightarrow}}

\def\ex2f{f(\mathbf{x}^*)}

\input{assignment.sty}
\begin{document}

\setassignment
\setduedate{Thursday, 17 October 2024, 23:59}

\serieheader
{Discrete Mathematics}
{2024}
{%
\textbf{Student:} Jeferson Morales Mariciano 
\href{mailto:jmorale@ethz.ch}{\(<\)jmorale@ethz.ch\(>\)} \\\\}
{\vspace{-1cm}}%\textbf{Discussed with:} }
{Assignment 4}{}

%----------------------------------------------------------------------------------
\section*{Exercise 4.5, Proving/Disproving Set Properties (\(\star\star\)) \hfill (8 Points)}

Prove or disprove the following statements.

\begin{enumerate}[label=\textbf{\alph*)}]
    \item For any sets \( A, B, C \) it holds
        \( 
            \left( 
                A \cup 
                \left( B \setminus C \right)
            \right) 
            \cap
            \left( B \cap C \right) 
            = \varnothing 
        \)
    \item For any sets \( A, B, C \) it holds
        \( 
            A \cap \left( B \setminus C \right)
            =
            \left( A \cap B \right) 
            \setminus 
            \left( 
                \left( A \cap B \right)
                \cap C 
            \right)
        \)
    \item For any sets \( A, B \) it holds
        \(
            |
            \mathcal{P}\left(
                \mathcal{P}\left( B \right)
                \right)
            |
            \geq 2
        \)
\end{enumerate}

\noindent \textbf{Expectation}: 
Argue using the definitions of 
\( \subseteq, \cup, \cap, |\cdot|,\mathcal{P}(\cdot), \setminus, \times \)
from the lecture notes.
You are allowed to use any results you have already seen in the lecture, 
including facts from Chapter 2 (e.g. the rules of Lemma 2.1), 
as well as \( F \lor \bot \equiv F \) and \( F \land \top \equiv F\). 
You can apply several rules/results in one step, 
but have to clearly state which rules/results you apply. 
To disprove a statement, provide a concrete counterexample.



\end{document}

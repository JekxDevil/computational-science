\documentclass[unicode,11pt,a4paper,oneside,numbers=endperiod,openany]{scrartcl}

% Required package
\usepackage{amssymb}
\usepackage{graphicx}
\usepackage{amsmath}
\usepackage{matlab-prettifier}
\usepackage{float}
\usepackage[export]{adjustbox}
\usepackage{multirow}
\usepackage{booktabs}
\usepackage{amsthm} % math theorems
\usepackage{ifthen}
\usepackage{physics} % responsive norm, abs, ...
\usepackage{algorithm}
\usepackage{algpseudocode}
\usepackage{array} % for table
\usepackage{mathtools} % for \coloneq
\usepackage{xcolor, soul} % highlight

\renewcommand{\thesubsection}{\arabic{subsection}}

\newtheorem{theorem}{Theorem}[section]
% 1: title, 2: label, 3: content
\newcommand{\myth}[3]{
    \begin{theorem}[#1] 
        \label{#2} 
        #3 
    \end{theorem}
}

% 1: color, 2: content
\newcommand{\mathcolorbox}[2]{\colorbox{#1}{\(\displaystyle #2\)}}

% 1: if numbered equation, 2: label, 3: content
\newcommand{\myex}[3]{
    \ifthenelse{\equal{#1}{true}}{
        \begin{equation} \label{#2} \begin{aligned} #3 \end{aligned} \end{equation}
    }{
        \begin{equation*} \label{#2} \begin{aligned} #3 \end{aligned} \end{equation*}
    }
}

% vector shortcut
\newcommand{\myvec}[1]{\begin{bmatrix} #1 \end{bmatrix}}

% 1: letter of vector, 2: 
\newcommand\vibar[2]{\mathbf{\overline{#1}}_#2}

% 1: caption, 2: label, 3: trim, 4: figure within figures folder
\newcommand{\myfigure}[4]{
    \begin{figure}[htbp]
    \centering
    \caption{#1}
    \label{#2}
    \includegraphics[width=\paperwidth, trim=#3]{./figures/#4}
    \end{figure}
}

\def\ex2f{f(\mathbf{x}^*)}

\input{assignment.sty}
\begin{document}

\setassignment
\setduedate{Thursday, 3 October 2024, 23:59}

\serieheader
{Discrete Mathematics}
{2024}
{%
\textbf{Student:} Jeferson Morales Mariciano 
\href{mailto:jmorale@ethz.ch}{\(<\)jmorale@ethz.ch\(>\)} \\\\}
{\vspace{-1cm}}%\textbf{Discussed with:} }
{Assignment 2}{}

%----------------------------------------------------------------------------------
\section*{Exercise 2.3, Simplifying a Formula (\(\star\)) \hfill (8 Points)}

Consider the propositional formula

\[
F = \left( (B \lor C) \to ((A \lor \neg B) \land C) \right) \lor (A \land \neg C)
\]

\noindent Give a formula \( G \) that is equivalent to \( F \), but in which each atomic formula \( A \), \( B \), and \( C \)
appears at most once. Prove that \( F \equiv G \) by providing a sequence of equivalence transformations 
with \textit{at most} 12 steps.

\noindent \textbf{Expectation.} 
Your proof should be in the form of a sequence of steps, where each step
consists of applying the definition of \( \to \) 
(that is \( F \to G \equiv \neg F \lor G \)), 
one of the rules given in Lemma 2.1 of the lecture notes 
\footnote{Lemma 2.1 states rules involving propositional symbols, 
but you may apply those rules at the level of formulas 
(see Section 2.3.5 of the lecture notes).}, 
or one of the following rules: 
\( F \land \neg F \equiv \bot \), 
\( F \land \bot \equiv \bot \), 
\( F \lor \bot \equiv F \), 
\( F \lor \neg F \equiv \top \), 
\( F \land \top \equiv F \), 
and \( F \lor \top \equiv \top \). 
For this exercise, associativity is to be applied as in Lemma 2.1.3. 
Each step of your proof should apply a \textit{single} rule \textit{once} 
and state \textit{which} rule was applied.
\\

The formula \( G \) equivalent to \( F \) is given by \( G = A \lor \neg B \), 
where each propositional symbol appears at most once with \( C \) not comparing at all.
The proof is presented below: 
it is 12-steps long as requested and follows a sequence of equivalence transformations.

\myex{true}{ex2-3-proof}{
F &= \left( 
        \left(B \lor C \right) 
        \rightarrow 
        \left( 
            \left(A \lor \neg B \right) 
            \land C 
        \right) 
    \right) 
    \lor \left( A \land \neg C \right) \\
&\equiv \left( 
        \mathcolorbox{yellow}{
            \neg \left(B \lor C \right) \lor \left( \left( A \lor \neg B \right) \land C \right)
        } 
    \right) 
    \lor \left( A \land \neg C \right) 
&&&&&&&&&& \text{def. of implication \(\rightarrow\)} \\
&\equiv \left( 
        \left(
            \mathcolorbox{yellow}{ \neg B \land \neg C} 
        \right) 
        \lor 
        \left( 
            \left(A \lor \neg B \right) 
            \land C 
        \right) 
    \right) 
    \lor \left( A \land \neg C \right)
&&&&&&&&&& \text{def. De Morgan rule} \\
&\equiv \left( 
    \mathcolorbox{yellow}{
        \left( 
            \left(A \lor \neg B \right) 
            \land C 
        \right) 
        \lor 
        \left( \neg B \land \neg C \right) 
    }
    \right) 
    \lor \left( A \land \neg C \right)
&&&&&&&&&& \text{def. commutativity of } \lor \\
&\equiv  
    \left( 
        \left(A \lor \neg B \right) 
        \land C 
    \right) 
    \lor 
    \left(
    \mathcolorbox{yellow}{
        \left( \neg B \land \neg C \right) 
        \lor 
        \left( A \land \neg C \right)
    }
    \right) 
&&&&&&&&&& \text{def. associativity of } \lor \\
&\equiv  
    \left( 
        \left(A \lor \neg B \right) 
        \land C 
    \right) 
    \lor 
    \left(
        \left( \neg B \land \neg C \right) 
        \lor 
        \left( \mathcolorbox{yellow}{\neg C \land A} \right)
    \right) 
&&&&&&&&&& \text{def. commutativity of } \land \\
&\equiv  
    \left( 
        \left(A \lor \neg B \right) 
        \land C 
    \right) 
    \lor 
    \left(
        \left( \mathcolorbox{yellow}{\neg C \land \neg B} \right) 
        \lor 
        \left( \neg C \land A \right)
    \right) 
&&&&&&&&&& \text{def. commutativity of } \land \\
&\equiv  
    \left( 
        \left(A \lor \neg B \right) 
        \land C 
    \right) 
    \lor 
    \left(
        \mathcolorbox{yellow}{
            \neg C 
            \land \left( \neg B \lor A \right) 
        }
    \right) 
&&&&&&&&&& \text{def. 1st distributivity law} \\
&\equiv  
    \left( 
        \left(A \lor \neg B \right) 
        \land C 
    \right) 
    \lor 
    \left(
        \neg C 
        \land \left( \mathcolorbox{yellow}{A \lor \neg B} \right) 
    \right) 
&&&&&&&&&& \text{def. commutativity of } \lor \\
&\equiv  
    \left( 
        \left(A \lor \neg B \right) 
        \land C 
    \right) 
    \lor 
    \left(
    \mathcolorbox{yellow}{
        \left( A \lor \neg B \right) 
        \land 
        \neg C 
    }
    \right) 
&&&&&&&&&& \text{def. commutativity of } \land \\
&\equiv  
    \mathcolorbox{yellow}{
    \left(A \lor \neg B \right) 
    \land 
    \left( C \lor \neg C \right) 
    }
&&&&&&&&&& \text{def. 1st distributivity law} \\
&\equiv  
    \left(A \lor \neg B \right) 
    \land 
    \mathcolorbox{yellow}{
    \top 
    }
&&&&&&&&&& F \lor \neg F \equiv \top \\
&\equiv  
    A \lor \neg B
&&&&&&&&&& F \land \top \equiv F \\
}

\end{document}

\documentclass[unicode,11pt,a4paper,oneside,numbers=endperiod,openany]{scrartcl}

% Required package
\usepackage{amssymb}
\usepackage{graphicx}
\usepackage{amsmath}
\usepackage{matlab-prettifier}
\usepackage{float}
\usepackage[export]{adjustbox}
\usepackage{multirow}
\usepackage{booktabs}
\usepackage{amsthm} % math theorems
\usepackage{ifthen}
\usepackage{physics} % responsive norm, abs, ...
\usepackage{algorithm}
\usepackage{algpseudocode}
\usepackage{array} % for table
\usepackage{mathtools} % for \coloneq
\usepackage{xcolor, soul} % highlight
\usepackage{enumitem} % custom enumeration in lists

\renewcommand{\thesubsection}{\arabic{subsection}}

\newtheorem{theorem}{Theorem}[section]
% 1: title, 2: label, 3: content
\newcommand{\myth}[3]{
    \begin{theorem}[#1] 
        \label{#2} 
        #3 
    \end{theorem}
}

% 1: color, 2: content
\newcommand{\mathcolorbox}[2]{\colorbox{#1}{\(\displaystyle #2\)}}

% 1: if numbered equation, 2: label, 3: content
\newcommand{\myex}[3]{
    \ifthenelse{\equal{#1}{true}}{
        \begin{equation} \label{#2} \begin{aligned} #3 \end{aligned} \end{equation}
    }{
        \begin{equation*} \label{#2} \begin{aligned} #3 \end{aligned} \end{equation*}
    }
}

% vector shortcut
\newcommand{\myvec}[1]{\begin{bmatrix} #1 \end{bmatrix}}

% 1: letter of vector, 2: 
\newcommand\vibar[2]{\mathbf{\overline{#1}}_#2}

% 1: caption, 2: label, 3: trim, 4: figure within figures folder
\newcommand{\myfigure}[4]{
    \begin{figure}[htbp]
    \centering
    \caption{#1}
    \label{#2}
    \includegraphics[width=\paperwidth, trim=#3]{./figures/#4}
    \end{figure}
}

% proof step
\newcommand{\pstep}{\overset{.}{\Longrightarrow}}

\def\ex2f{f(\mathbf{x}^*)}

\input{assignment.sty}
\begin{document}

\setassignment
\setduedate{Thursday, 10 October 2024, 23:59}

\serieheader
{Discrete Mathematics}
{2024}
{%
\textbf{Student:} Jeferson Morales Mariciano 
\href{mailto:jmorale@ethz.ch}{\(<\)jmorale@ethz.ch\(>\)} \\\\}
{\vspace{-1cm}}%\textbf{Discussed with:} }
{Assignment 3}{}

%----------------------------------------------------------------------------------
\section*{Exercise 3.2, From Natural Language to a Formula (\(\star\)) \hfill (4 Points)}

Consider the universe \(U = \mathbb{N} \setminus\{0\}\). 
Express each of the following statements with a formula in predicate logic, 
in which the only predicates appearing are 
\(divides(x, y)\), \(equals(x, y)\) and \(prime(x)\) 
(instead of divides(x, y) and equals(x, y) you can write \(x | y\) and \(x = y\) accordingly). 
You can also use the symbols \(+\) and \(\cdot\) 
to denote the addition and multiplication functions, 
and you can use constants (e.g., 0, 1, . . .). 
You can also use \(\leftrightarrow\). 
No justification is required.

\begin{enumerate}[label=(\roman*)]
    \item \((\star)\) If a number divides two numbers, then it also divides their sum.
    \item \((\star)\) The only divisors of a prime number are 1 and the number itself.
    \item \((\star)\) 1 is the only natural number which has an inverse.
    \item \((\star)\) A prime number divides the product of two natural numbers 
        if and only if it divides at least one of them.
\end{enumerate}

\textbf{i)}
\myex{true}{ex3-2-i}{
    \forall x \; \forall y \; \forall z \; %\in U=\mathbb{N} \setminus \{0\} 
    \left( 
        \left(
            \left( x \, | \, y \right) 
            \land 
            \left( x \, | \, z \right) 
        \right)
        \rightarrow 
        \left( x \, | \left( y + z \right) \right) 
    \right)
}

\textbf{ii)}
\myex{true}{ex3-2-ii}{
    \forall x \; \forall y \; %\in U=\mathbb{N} \setminus \{0\} 
    \left( 
        \left(
            prime(x)
            \land 
            \left( y \, | \, x \right) 
        \right)
        \rightarrow 
        \left(
            \left( y = 1 \right) 
            \lor
            \left( y = x \right)
        \right)
    \right)
}

\textbf{iii)}
\myex{true}{ex3-2-iii}{
    \forall x \; \forall y \; %\in U=\mathbb{N} \setminus \{0\} 
    \left( 
        \left(
            x \cdot y = 1
        \right)
        \rightarrow 
        \left(
            \left( x = 1 \right) 
            \land
            \left( y = 1 \right)
        \right)
    \right)
}

\textbf{iv)}
\myex{true}{ex3-2-iv}{
    \forall x \;  %\in U=\mathbb{N} \setminus \{0\} 
    \left( 
        prime(x)
        \rightarrow 
        \forall y \; \forall z \;
        \left(
            \left( x \, | \left( y \cdot z \right) \right) 
            \longleftrightarrow
            \left(
                \left( x \, | \, y \right)
                \lor
                \left( x \, | \, z \right)
            \right)
        \right)
    \right)
}


\section*{Exercise 3.8, Proof by Contradiction (\(\star\)) \hfill (4 Points)}

Let \(n, m \in \mathbb{N}\) be arbitrary. 
We say “n divides m” and write \(n | m\) 
if there exists a \(k \in \mathbb{N}\) such that \(k \cdot n = m\). 
Prove that the following statement is true, 
using a proof by contradiction:
\[
n | m \;\text{ and }\; n |(m + 1) \Longrightarrow n = 1.
\]
You are allowed to invoke the statement 3.2 \textbf{iii)} from above to justify one step.
You must use the same notation as in the lecture notes, 
i.e. precisely state what your statements S and T are, 
and justify each of your proof steps.
\\

\begin{proof}
    From Definition 2.17, a proof by contradiction of a statement S proceeds in 3 steps:
    \begin{enumerate}
        \item Find a suitable mathematical statement \( T \).
        \item Prove that \( T \) is false.
        \item Assume that S is false and prove (from this assumption) 
            that T is true (a contradiction).
    \end{enumerate}

    The general informal overview of the proof is to use the statement 3.2 
    \textbf{iii)}~\ref{ex3-2-iii} 
    i.e. the only natural number which has an inverse is 1.
    It will be used to prove that statement T is false,
    namely that there exist some pair(s) of number belonging to \( \mathbb{N} \) 
    such that their product is \( 1 \) and at least one of them is not \( 1 \),
    meaning that 1 is not the only natural number with an inverse.
    Then, from assuming that \( \neg S \) is true, 
    we will prove that T is true through composition of implications, 
    which will lead to a contradiction. 
    The statement provided to be proved will be our statement S,
    namely that if a number \( n \) divides two other consecutive numbers, 
    then it must be equal to \( 1 \).


    The interpretation is fixed 
    by remarking that the universe is \( U = \mathbb{N} \) of natural numbers, 
    the predicates will be \( divides(x, y) \), \( equals(x, y) \), 
    \( sum(x,y) \), \( multiplication(x,y) \).
    Finally, the function symbols used will be \( | \,, = \,, + \,, \cdot \, \) 
    for divisibility, equality, addition and multiplication, respectively.
    Nevertheless, we stick to make mathematical statements about formulas.

    From the statement 3.2 \textbf{iii)}~\ref{ex3-2-iii}: 
    \myex{false}{ex8-iii}{
        \forall x \; \forall y \; %\in U=\mathbb{N} \setminus \{0\} 
        \left( 
            \left(
                x \cdot y = 1
            \right)
            \rightarrow 
            \left(
                \left( x = 1 \right) 
                \land
                \left( y = 1 \right)
            \right)
        \right)
    }

    its negation leads to state that there exists some pair of numbers which product is 1
    and at least one of these numbers is not 1, 
    meaning 1 is not the only natural number with an inverse as shown below:

    \myex{false}{ex8-t-construction}{
        & \neg \forall x \; \forall y \; %\in U=\mathbb{N} \setminus \{0\} 
        \left( 
            \left(
                x \cdot y = 1
            \right)
            \rightarrow 
            \left(
                \left( x = 1 \right) 
                \land
                \left( y = 1 \right)
            \right)
        \right)
        \\
        \pstep \quad & \exists x \; \neg \forall y \; %\in U=\mathbb{N} \setminus \{0\} 
        \left( 
            \left(
                x \cdot y = 1
            \right)
            \rightarrow 
            \left(
                \left( x = 1 \right) 
                \land
                \left( y = 1 \right)
            \right)
        \right)
        \\
        \pstep \quad & \exists x \; \exists y \; %\in U=\mathbb{N} \setminus \{0\} 
        \neg \left( 
            \left(
                x \cdot y = 1
            \right)
            \rightarrow 
            \left(
                \left( x = 1 \right) 
                \land
                \left( y = 1 \right)
            \right)
        \right)
        \\
        \pstep \quad & \exists x \; \exists y \; %\in U=\mathbb{N} \setminus \{0\} 
        \left( 
            \left(
                x \cdot y = 1
            \right)
            \land
            \neg \left(
                \left( x = 1 \right) 
                \land
                \left( y = 1 \right)
            \right)
        \right)
        \\
        \pstep \quad & \exists x \; \exists y \; %\in U=\mathbb{N} \setminus \{0\} 
        \left( 
            \left(
                x \cdot y = 1
            \right)
            \land
            \left(
                \neg \left( x = 1 \right) 
                \lor
                \neg \left( y = 1 \right)
            \right)
        \right)
    }

    Hence, w.r.t. interpretation the clearly false statement T is 
    "1 is not the only natural number which has an inverse".
    Let \( n, k \in U \) 
    \[ 
        T \pstep \;\;
         \exists n \; \exists k \; %\in U=\mathbb{N} \setminus \{0\} 
        \left( 
            \left(
                n \cdot k = 1
            \right)
            \land
            \left(
                \neg \left( n = 1 \right) 
                \lor
                \neg \left( k = 1 \right)
            \right)
        \right)
    \]

    Now, the statement to be proved S is 
    "given n dividing two consecutive numbers, then it must be equal to 1":

    \[
        S 
        \pstep \forall n \; \forall m \; 
        \left(
            \left(
                \left( n \, | \, m \right) 
                \land  
                \left( n \, | \left( m + 1 \right) \right) 
            \right)
            \longrightarrow 
            n = 1
        \right)
    \]

    Assuming that the statement S is false, 
    meaning that there exist some \( n, m \) such that 
    \( n \) divides \( m \) and \( n \) divides \( m+1 \), and \( n \) is not equal to 1:

    \myex{false}{ex8-s-false}{
        \neg S  
        \pstep \quad & 
        \neg \forall n \; \forall m \; 
        \left(
            \left(
                \left( n \, | \, m \right) 
                \land  
                \left( n \, | \left( m + 1 \right) \right) 
            \right)
            \longrightarrow 
            n = 1
        \right)
        \\
        \pstep \quad & 
        \exists n \; \neg \forall m \; 
        \left(
            \left(
                \left( n \, | \, m \right) 
                \land  
                \left( n \, | \left( m + 1 \right) \right) 
            \right)
            \longrightarrow 
            n = 1
        \right)
        \\
        \pstep \quad & 
        \exists n \; \exists m \; 
        \neg \left(
            \left(
                \left( n \, | \, m \right) 
                \land  
                \left( n \, | \left( m + 1 \right) \right) 
            \right)
            \longrightarrow 
            n = 1
        \right)
        \\
        \pstep \quad & 
        \exists n \; \exists m \; 
        \neg \left(
            \neg \left(
                \left( n \, | \, m \right) 
                \land  
                \left( n \, | \left( m + 1 \right) \right) 
            \right)
            \lor
            \left( n = 1 \right)
        \right)
        \\
        \pstep \quad & 
        \exists n \; \exists m \; 
        \left(
            \neg \neg \left(
                \left( n \, | \, m \right) 
                \land  
                \left( n \, | \left( m + 1 \right) \right) 
            \right)
            \land
            \neg \left( n = 1 \right)
        \right)
        \\
        \pstep \quad & 
        \exists n \; \exists m \; 
        \left(
            \left(
                \left( n \, | \, m \right) 
                \land  
                \left( n \, | \left( m + 1 \right) \right) 
            \right)
            \land
            \neg \left( n = 1 \right)
        \right)
    }

    from definition of divisibility we have that:
    
    \myex{false}{ex8-def-divisibility}{
        n \, | \, m 
        \; \overset{\text{def}}{\Longleftrightarrow} \;
        \exists k \; \left( k \cdot n = m \right)
    }

    so we can rewrite the statements \( n \, | \, m \) and \( n \, | \, \left(m+1\right) \) as:

    \myex{false}{ex8-rewrite-nm}{
        n \, | \, m 
        \pstep \quad &     
        k_1 \cdot n = m
        \\
        n \, | \left(m+1\right)
        \pstep \quad &
        k_2 \cdot n = m + 1
        \; \pstep \; 
        \left( k_2 \cdot n \right) - 1 = m
    }

    since \( m = m \), we can inject the first equation into the second one:

    \myex{false}{ex8-m-inject}{
        & k_1 \cdot n = \left( k_2 \cdot n \right) - 1
        \\ \pstep \quad &
        \left( k_1 \cdot n \right) - \left( k_2 \cdot n \right) = -1
        \\ \pstep \quad &
        n \cdot \left( k_1 - k_2 \right) = -1
        \\ \pstep \quad &
        n \cdot \left( k_2 - k_1 \right) = 1
        \\ \pstep \quad &
        n \cdot k = 1
    }

    By plugging all back into the formula derived from assumption of \( \neg S \), 
    using composition of implications, 
    we get a contradiction that \( n \) time \( k \) gives 1, meaning it has an inverse, 
    but \( n \) is not equal to \( 1 \), 
    meaning that there exists a number that has an inverse and is not 1. 
    It corresponds to the form of the statement T already proved to be false,
    as shown below:

    \myex{false}{ex8-T-contradiction}{
        \exists n \; \exists m \; \exists k \;
        \left(
            \left(
                \left( n \, | \, m \right) 
                \land  
                \left( n \, | \left( m + 1 \right) \right) 
            \right)
            \land
            \underbrace{
            \neg \left( n = 1 \right)
            \land
            \left( \left( k \cdot n \right) = 1 \right)
            }_{T \text{ false for arbitrary k and n}}
        \right)
    }

    Therefore, the statement \( S \) must be true, it is satisfiable 
    and holds for our interpretation, concluding the proof. 
\end{proof}

\end{document}

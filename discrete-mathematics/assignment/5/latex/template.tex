\documentclass[unicode,11pt,a4paper,oneside,numbers=endperiod,openany]{scrartcl}

% Required package
\usepackage{amssymb}
\usepackage{graphicx}
\usepackage{amsmath}
\usepackage{matlab-prettifier}
\usepackage{float}
\usepackage[export]{adjustbox}
\usepackage{multirow}
\usepackage{booktabs}
\usepackage{amsthm} % math theorems
\usepackage{ifthen}
\usepackage{physics} % responsive norm, abs, ...
\usepackage{algorithm}
\usepackage{algpseudocode}
\usepackage{array} % for table
\usepackage{mathtools} % for \coloneq
\usepackage{xcolor, soul} % highlight
\usepackage{enumitem} % custom enumeration in lists
\usepackage{dsfont} % 1 for identity matrix
\usepackage{blkarray, bigstrut} % block matrix with labels
\usepackage{tikz} % for drawing

\newcommand\topstrut[1][1.2ex]{\setlength\bigstrutjot{#1}{\bigstrut[t]}}
\newcommand\botstrut[1][0.9ex]{\setlength\bigstrutjot{#1}{\bigstrut[b]}}

\usetikzlibrary{positioning}
\tikzset{
    > = stealth,
    every path/.append style = {
        arrows = ->,
    }
}

\renewcommand{\thesubsection}{\arabic{subsection}}

\newtheorem{theorem}{Theorem}[section]
% 1: title, 2: label, 3: content
\newcommand{\myth}[3]{
    \begin{theorem}[#1] 
        \label{#2} 
        #3 
    \end{theorem}
}

% 1: color, 2: content
\newcommand{\mathcolorbox}[2]{\colorbox{#1}{\(\displaystyle #2\)}}

% 1: if numbered equation, 2: label, 3: content
\newcommand{\myex}[3]{
    \ifthenelse{\equal{#1}{true}}{
        \begin{equation} \label{#2} \begin{aligned} #3 \end{aligned} \end{equation}
    }{
        \begin{equation*} \label{#2} \begin{aligned} #3 \end{aligned} \end{equation*}
    }
}

% vector shortcut
\newcommand{\myvec}[1]{\begin{bmatrix} #1 \end{bmatrix}}
\newcommand{\mymat}[1]{\begin{pmatrix} #1 \end{pmatrix}}

% 1: letter of vector, 2: 
\newcommand\vibar[2]{\mathbf{\overline{#1}}_#2}

% 1: caption, 2: label, 3: trim, 4: figure within figures folder
\newcommand{\myfigure}[4]{
    \begin{figure}[htbp]
    \centering
    \caption{#1}
    \label{#2}
    \includegraphics[width=\paperwidth, trim=#3]{./figures/#4}
    \end{figure}
}

% proof step
\newcommand{\pstep}{\overset{.}{\Longrightarrow}}

\def\ex2f{f(\mathbf{x}^*)}

\input{assignment.sty}
\begin{document}

\setassignment
\setduedate{Thursday, 24 October 2024, 23:59}

\serieheader
{Discrete Mathematics}
{2024}
{%
\textbf{Student:} Jeferson Morales Mariciano 
\href{mailto:jmorale@ethz.ch}{\(<\)jmorale@ethz.ch\(>\)} \\\\}
{\vspace{-1cm}}%\textbf{Discussed with:} }
{Assignment 5}{}

%----------------------------------------------------------------------------------
\section*{Exercise 5.5, Properties of Relations (\(\star\)) \hfill (8 Points)}

Prove or disprove the following claims:

\begin{enumerate}[label=\textbf{\alph*)}]
    \item A relation \( \rho \) on a set \( A \) is symmetric on \( A \) 
        if and only if \( \rho^2 \) is symmetric on \( A \).

    \item If \( \rho \) is a relation on a set \( A \) 
        that is symmetric and antisymmetric,
        then it must hold \( \rho = \textsf{id}_A \).

    \item Define the relations \( \rho_1 \) and \( rho_2 \) on \( \mathbb{Z} \) as
    \[
        a \, \rho_1 \, b \iff b = a + 1, \qquad a \, \rho_2 \, b \iff b \equiv_2 a.
    \]
    Then for \( \rho = \rho_1 \cup \rho_2 \) it holds \( \rho^2 = \mathbb{Z} \times \mathbb{Z} \).
\end{enumerate}


\subsection*{a)}

The claim is false, a counterexample follows:

\myex{false}{ex5-5-a-counterexample}{
    \rho = \{ (a, c), (b, d), (c, b), (d, a) \}
    \\
    \rho^2 = \{ (a, b), (b, a), (c, d), (d, c) \}
}

The relation \( \rho^2 \) is symmetric, but \( \rho \) is not, 
which disprove the claim if and only if (\( \iff \)) from the right to left part 
(\( \Longleftarrow \)). 

The intuition behind the counterexample is given by expanding the right hand side (\( \Longleftarrow \))  
of the claim into a composition of implications as follows. 
Note the universe \( \mathcal{U} \) is an arbitrary set \( A \) of elements.

\myex{false}{ex5-5-a-proof}{
    & \rho^2 \text{ symmetric on } A 
    \\ & \pstep \; 
    \forall a \forall b \left( a \, \rho^2 \, b \Longleftrightarrow b \, \rho^2 \, a \right)
    \hspace{1cm} & \text{(def 3.15 symmetry)}
    \\ & \pstep \; 
    \forall a \forall b \left( (a, b) \in \rho^2 \land (b,a) \in \rho^2 \right)
    \hspace{1cm} & \text{(def 3.9 relation)}
    \\ & \pstep \; 
    \forall a \forall b \left( 
        \exists c \left( (a, c) \in \rho \land (c,b) \in \rho \right)
        \land
        \exists d \left( (b, d) \in \rho \land (d,a) \in \rho \right)
    \right)
    \hspace{1cm} & \text{(def 3.12 composition)}
}

So it is sufficient to find a relation \( \rho \) that is not symmetric, 
i.e. which have \( c \neq d \), satisfying the constraint of the composition of implications above. 
A visualization is provided in the following matrix:

\myex{false}{ex5-5-a-matrix}{
    M^{\rho^2} = 
    \begin{blockarray}{l c c c c}
        & a & b & c & d \\
    \begin{block}{l [c c c c]}
        a & 0 & 1 & 0 & 0 \topstrut \\
        b & 1 & 0 & 0 & 0 \\
        c & 0 & 0 & 0 & 1 \\
        d & 0 & 0 & 1 & 0 \botstrut \\
    \end{block}
    \end{blockarray}
    =
    M^{\rho} \cdot M^{\rho}
    \,,\quad \text{ where } \quad
    M^{\rho} =
    \begin{blockarray}{l c c c c}
        & a & b & c & d \\
    \begin{block}{l [c c c c]}
        a & 0 & 0 & 1 & 0 \topstrut \\
        b & 0 & 0 & 0 & 1 \\
        c & 0 & 1 & 0 & 0 \\
        d & 1 & 0 & 0 & 0 \botstrut \\
    \end{block}
    \end{blockarray}
}

The matrices resemble the counterexample, where \( \rho^2 \) is symmetric,
while \( \rho \) is clearly not, 
i.e. \( M^{\rho} \neq (M^{\rho})^{\intercal} \iff \rho \neq \hat{\rho} \).

A graph representation is provided for the sake of visualization, 
where the composition of the same structure choosen by the matrices is visualize.
The elements of \( \rho^2 \) are given by choosing a path with length 2,
meaning a green edge and the subsequent black one on the right.
Doing so for every vertex gives the \( \rho^2 \) relation. \\

% left graph for rho
\begin{minipage}{0.4\textwidth}
\centering
graph \( \rho \) \\
\vspace{0.2cm}
\tikz{
    \node (a) {$a$};
    \node (b) [below = of a] {$b$};
    \node (c) [below = of b] {$c$};
    \node (d) [below = of c] {$d$};

    \path (a) edge [bend left=10, in=240] (c);
    \path (c) edge (b);
    \path (b) edge [bend left=30] (d);
    \path (d) edge [bend left=20, in=135] (a);
}
\end{minipage}
%
% right graph for rho composition
\begin{minipage}{0.4\textwidth}
\centering
graph of \( \rho^2 \) \\
\vspace{0.2cm}
\tikz{
    \node (al) {$a$};
    \node (bl) [below = of al] {$b$};
    \node (cl) [below = of bl] {$c$};
    \node (dl) [below = of cl] {$d$};
    \node (ar) [right = of al] {$a$};
    \node (br) [below = of ar] {$b$};
    \node (cr) [below = of br] {$c$};
    \node (dr) [below = of cr] {$d$};

    \path (al) edge [green!70!black] (cr);
    \path (bl) edge [green!70!black] (dr);
    \path (cl) edge [green!70!black] (br);
    \path (dl) edge [green!70!black] (ar);

    \path (ar) edge [bend left=10, in=240] (cr);
    \path (cr) edge (br);
    \path (br) edge [bend left=30] (dr);
    \path (dr) edge [bend left=20, in=135] (ar);
}
\end{minipage}


%-----------------------------------------------------------------------------------
\subsection*{b)}

The claim is false, a counterexample follows:

\myex{false}{ex5-5-b-counterexample}{
    \rho = \emptyset \neq \textsf{id}_A
}

The relation \( \rho \) is both symmetric and antisymmetric.
However, \( \rho \) is not the identity relation as the claim states.

The satisfiability conditions for the relation 
being both symmetric and antisymmetric are explained as follows.
The matrix for the counterexample is the all-zeros matrix \( \mathbf{0} \) given by:

\myex{false}{ex5-5-b-matrix}{
    M^{\rho} = 
    \mathbf{0} =
    \myvec{
        0       & \cdots    & 0         \\
        \vdots  & 0         & \vdots    \\
        0       & \cdots    & 0
    }
}

It is clear that the relation is symmetric by Definition 3.15,
\( \forall a \forall b \in A \;\; a \, \rho \, b \iff b \, \rho \, a \),
i.e. the matrix is symmetric 
\( M^{\rho} = (M^{\rho})^{\intercal} \iff \rho = \hat{\rho} \).

Moreover, the relation is antisymmetric by Definition 3.16,
\( \forall a \forall b \in A \;\; a \, \rho \, b \, \land \, b \, \rho \, a \implies a = b \),
i.e. the intersection between the relation and its inverse is a subset of the indentity relation,
\( \rho \cap \hat{\rho} \subseteq \textsf{id}_A \).
Notice that for the counterexample, the intersection is the empty set,
which is a subset of every set by Lemma 3.6, including the identity relation.

\myex{false}{ex5-5-b-antisymmetric}{
    \rho = \emptyset \subseteq \left( A \times A \right),
    \quad 
    \rho \cap \hat{\rho} = \emptyset \subseteq \textsf{id}_{A}
    \quad \because \quad 
    \emptyset \in \mathcal{P(\mathsf{id}_A)}
}

In general symmetry and antisymmetry are independent properties of a relation.

%-----------------------------------------------------------------------------------
\subsection*{c)}

The claim is true, and it can be proven as follows:

we will decompose the claim through composition of implications from which will lead
to a clear satisfiability problem that show is satisfiable using a proof by case distinction.

\subsubsection*{Composition of Implications}

\subsubsection*{Case Distinction}

Check that the statement is always satisfiable for all cases of \( a, b \),
i.e. that there always exist some suitable \( c \) in the universe to satisfy the claim.

The cases are represented by the following matrix:

\[
    \begin{array}{|cc|c|}
        \hline
        a & b & c \\
        \hline
        0 & 0 & 0    \\
        0 & 1 & s(a) \\
        1 & 0 & s(a) \\
        1 & 1 & 1    \\
        \hline
    \end{array}
\]

where \( s(x) \) is the successor of \( x \) in the universe.
Finally, it follows that the claim is always satisfiable, 
meaning is true for all cases of \( a, b \in \mathcal{U} = \mathbb{Z} \times \mathbb{Z} \).
Concluding, \( \rho^2 = \mathds{1} = \mathbb{Z} \times \mathbb{Z} \).


\end{document}

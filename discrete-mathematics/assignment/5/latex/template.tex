\documentclass[unicode,11pt,a4paper,oneside,numbers=endperiod,openany]{scrartcl}

% Required package
\usepackage{amssymb}
\usepackage{graphicx}
\usepackage{amsmath}
\usepackage{matlab-prettifier}
\usepackage{float}
\usepackage[export]{adjustbox}
\usepackage{multirow}
\usepackage{booktabs}
\usepackage{amsthm} % math theorems
\usepackage{ifthen}
\usepackage{physics} % responsive norm, abs, ...
\usepackage{algorithm}
\usepackage{algpseudocode}
\usepackage{array} % for table
\usepackage{mathtools} % for \coloneq
\usepackage{xcolor, soul} % highlight
\usepackage{enumitem} % custom enumeration in lists
\usepackage{dsfont} % 1 for identity matrix

\renewcommand{\thesubsection}{\arabic{subsection}}

\newtheorem{theorem}{Theorem}[section]
% 1: title, 2: label, 3: content
\newcommand{\myth}[3]{
    \begin{theorem}[#1] 
        \label{#2} 
        #3 
    \end{theorem}
}

% 1: color, 2: content
\newcommand{\mathcolorbox}[2]{\colorbox{#1}{\(\displaystyle #2\)}}

% 1: if numbered equation, 2: label, 3: content
\newcommand{\myex}[3]{
    \ifthenelse{\equal{#1}{true}}{
        \begin{equation} \label{#2} \begin{aligned} #3 \end{aligned} \end{equation}
    }{
        \begin{equation*} \label{#2} \begin{aligned} #3 \end{aligned} \end{equation*}
    }
}

% vector shortcut
\newcommand{\myvec}[1]{\begin{bmatrix} #1 \end{bmatrix}}

% 1: letter of vector, 2: 
\newcommand\vibar[2]{\mathbf{\overline{#1}}_#2}

% 1: caption, 2: label, 3: trim, 4: figure within figures folder
\newcommand{\myfigure}[4]{
    \begin{figure}[htbp]
    \centering
    \caption{#1}
    \label{#2}
    \includegraphics[width=\paperwidth, trim=#3]{./figures/#4}
    \end{figure}
}

% proof step
\newcommand{\pstep}{\overset{.}{\Longrightarrow}}

\def\ex2f{f(\mathbf{x}^*)}

\input{assignment.sty}
\begin{document}

\setassignment
\setduedate{Thursday, 24 October 2024, 23:59}

\serieheader
{Discrete Mathematics}
{2024}
{%
\textbf{Student:} Jeferson Morales Mariciano 
\href{mailto:jmorale@ethz.ch}{\(<\)jmorale@ethz.ch\(>\)} \\\\}
{\vspace{-1cm}}%\textbf{Discussed with:} }
{Assignment 5}{}

%----------------------------------------------------------------------------------
\section*{Exercise 5.5, Properties of Relations (\(\star\)) \hfill (8 Points)}

Prove or disprove the following claims:

\begin{enumerate}[label=\textbf{\alph*)}]
    \item A relation \( \rho \) on a set \( A \) is symmetric on \( A \) 
        if and only if \( \rho^2 \) is symmetric on \( A \).

    \item If \( \rho \) is a relation on a set \( A \) 
        that is symmetric and antisymmetric,
        then it musta hold \( \rho = \textsf{id}_A \).

    \item Define the relations \( \rho_1 \) and \( rho_2 \) on \( \mathbb{Z} \) as
    \[
        a \, \rho_1 \, b \iff b = a + 1, \qquad a \, \rho_2 \, b \iff b \equiv_2 a.
    \]
    Then for \( \rho = \rho_1 \cup \rho_2 \) it holds \( \rho^2 = \mathbb{Z} \times \mathbb{Z} \).
\end{enumerate}


\subsection*{a)}

The claim is false, a counterexample follows:

\myex{false}{ex5-5-a-counterexample}{
    \rho = \{ (a, c), (b, d), (c, b), (d, a) \}
    \\
    \rho^2 = \{ (a, b), (b, a), (c, d), (d, c) \}
}

The relation \( \rho^2 \) is symmetric, but \( \rho \) is not, 
which disprove the claim if and only if (\( \iff \)) from the right to left part 
(\( \Longleftarrow \)). 

%-----------------------------------------------------------------------------------
\subsection*{b)}

The claim is false, a counterexample follows:

\myex{false}{ex5-5-b-counterexample}{
    \rho = \emptyset \neq \textsf{id}_A
}

The relation \( \rho \) is both symmetric and antisymmetric.
However, \( \rho \) is not the identity relation as the claim states.

%-----------------------------------------------------------------------------------
\subsection*{c)}

The claim is true, and it can be proven as follows:

we will decompose the claim through composition of implications from which will lead
to a clear satisfiability problem that show is satisfiable using a proof by case distinction.

\subsubsection*{Composition of Implications}

\subsubsection*{Case Distinction}

Check that the statement is always satisfiable for all cases of \( a, b \),
i.e. that there always exist some suitable \( c \) in the universe to satisfy the claim.

The cases are represented by the following matrix:

\[
    \begin{array}{|cc|c|}
        \hline
        a & b & c \\
        \hline
        0 & 0 & 0    \\
        0 & 1 & s(a) \\
        1 & 0 & s(a) \\
        1 & 1 & 1    \\
        \hline
    \end{array}
\]

where \( s(x) \) is the successor of \( x \) in the universe.
Finally, it follows that the claim is always satisfiable, 
meaning is true for all cases of \( a, b \in \mathcal{U} = \mathbb{Z} \times \mathbb{Z} \).
Concluding, \( \rho^2 = \mathds{1} = \mathbb{Z} \times \mathbb{Z} \).


\end{document}
